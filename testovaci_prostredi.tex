V následující kapitole si popíšeme něco málo o sestavě, která byla pro účely testování a srovnáváni výkonu virtualizačních nástrojů použita. Krom hardwarové výbavy testovacího stroje, což je základní minimum, které by mělo být uvedeno v každé práci zabývající se testováním výkonu určitých softwarových produktů, si také uvedeme operační systém a jeho přesnou verzi. Pro zvýšení vypovídající hodnoty následujících testů, zde uvedu i přesné verze jednotlivývh nástrojů použitých při testování.

\section{Testovací sestava}
\subsection{Základní deska}
Základem celého počítače je deska 880GMH/U3S3 od firmy ASRock. Důvodem pro její zvolení byla údajná podpora nejnovější rodiny AMD procesorů Bulldozer a také její podpora taktování. Během osazení této desky jednim z nových procesorů, se ukázalo, že podporu bulldozerů sice deska má. Bohužel až s nejnovější verzí BIOSu. Tu samozřejmě většina prodávaných kusů neměla. Bylo tedy nutno použít starší AMD procesor se socketem AM2+ a vyšší. A následně provést update BIOSu. Poté už vše fungovalo v pořádku.
\subsubsection{Přehled kompletních parametrů}
\begin{itemize}
  \item Podpora pro AM3+ procesory, až 8 jader
  \item 100\,\% pevných kondezátorů (delší životnost) 
  \item Podpora pro Dual Channel DDR3 2000(OC)
  \item Podpora ATI™ Hybrid CrossFireX™
  \item Integrovaná GPU AMD Radeon HD 4250 graphics, DX10.1 class iGPU, Shader Model 4.1
  \item Video výstupy : D-Sub, DVI-D a HDMI
  \item 2 x USB 3.0, 2 x SATA3, C.C.R. (Combo Cooler Retention Module)
  \item Podpora dalších technologií: AXTU, Graphical UEFI, Instant Boot, Instant Flash, APP Charger, SmartView
\end{itemize}

\subsection{Procesor}
Jak už jsme se výšše zmínil, použit byl procesor od firmy AMD se socketem AM3+ . Jednalo se o jeden z prvních modelů nové rodiny Bulldozer. Přesné označení procesoru je AMD FX-4100. Jedná se o procesor se čtyřmi jádry, kde každý z nich je v základním nastavení taktován na frekvenci 3,6 GHz (respektive 3,7 -- 3,8 v Turbo módu). Díky odemknutému násobiči je tento procesor přímo předurčen pro přetaktování. V době testů byl však ponechán na jeho základních frekvencí s vypnutým turbo boostem.

\subsubsection{Podrobné parametry}
\begin{itemize}
  \item Socket: AM3+
  \item L2 cache: 4 MB
  \item Frekvence: 3600 MHz
  \item Typ: AMD FX
  \item Model: FX-4100
  \item Počet jader: 4
  \item Operační módy: 32-bit a 64-bit
  \item L3 cache: 8 MB
  \item Výrobní proces: 32\,nm
  \item TDP: 95\,W
\end{itemize}

\subsection{Disky}
Počítačová sestava je vybavena dvěma pevnými disky jedním rychlým systémovým SSD diskem Verbatim o kapacitě 128\,GB, a druhým datovým SATA-II diskem Samsung o kapacitě 1\,GB.
\subsubsection{Parametry systémového disku}
\begin{itemize}
  \item Výrobce: Verbatim
  \item Model: SSD Black Edition
  \item Kapacita: 128\,GB
  \item Rozhraní: Serial ATA II
  \item Průměrný vyhledávací čas: 0.1\,ms
  \item Rychlost čtení: 270\,MB/s
  \item Rychlost zápisu: 225\,MB/s
  \item Rychlost přenosu dat: 3\,Gbit/s
  \item Speciální funkce: podpora TRIM a NCQ
\end{itemize}

\subsubsection{Parametry datového disku}
\begin{itemize}
  \item Výrobce: Samsung
  \item Model: SpinPoint F1
  \item Kapacita: 1000\,GB 
  \item Průměrný vyhledávací čas: 8.9\,ms  
  \item Rychlost otáčení ploten: 7200\,rpm
  \item Vyrovnávací paměť: 32\,MB
  \item Rozhraní: Serial ATA II
  \item Rychlost přenosu dat: 3\,Gbit/s
  \item Speciální funkce: NCQ
\end{itemize}

\subsection{Operační paměť}
Celkově počítač obsahuje čtyři paměťové moduly, kde každý má veliost 4\,GB a jsou spárovány po dvou, tak aby bylo využito podpory dual-channel. Dohromady tedy sestava disponuje 16\,GB RAM. Což je pro testovací účely víc než dostačující.
\section{Operační systém}
\subsection{Zvolená platforma}
První věcí, kterou bylo třeba při výběru OS rozhodnou byla platforma potažmo jádro operačního systému. Jelikož několik zvolených virtualizačních nástrojů podporuje pouze Linux. Nebyl důvod se moc rozmýšlet. Chvíli jsem uvažoval o některé variantě BSD systému, jelikož by zde šla teoreticky rozchodit většina zvolených nástrojů, a ty které by zde nešli, mají svoji alternativu. Ale nakonec jsem tuto myšlenku potlačil, jelikož Linux je rozšířenější, a dá se předpokládat, že pokud někdo bude některý ze zmiňovaných virtualizačních nástrojů nasazovat, tak to bude právě na linuxu.

\subsection{Volba distribuce}
Dalším krokem po vybrání linuxu jako hlavní platformy, na níž budu provádět veškeré testování, byla volba distribuce. Zde to již bylo o něco složitější, jelikož existuje skoro bezpočet distribucí, kde každá má své klady a zápory. Prvním rozhodnutím, které zúžilo možný počet kandidátů, byla snaha vybrat distribuci, poskytující dostatek dokumentace, popisující zprovoznění pokud možno co největšího počtu z vybraných virtualizačních nástrojů. Druhým kritériem byla co největší aktuálnost dostupných verzí jednotlivých softwarových komponent distribuce.

Důvodem proč jsem hledal distribuci s co nejnovějšími verzemi aplikací, byl fakt, že virtualizační nástroje se velmi rychle vyvíjí a já chci poskytnout co nejvěrohodnější výsledky těchto nástrojů v dnešní době. Což by mělo mít za výhodu aktuálnost mých výsledků i po delší době. Tento přístup má i svě stinné stránky. Nejnovější verze sebou krom nejnovějších vylepšení mohou nést i některé nové chyby, což může vést k určitým nestabilitám dané aplikace. Jelikož většina použitého software bude ve stabilních verzích, není toto riziko příliš vysoké.

Oba dva předchozí požadavky nejlépe splňovaly dve distribuce. Jednou znich bylo Ubuntu, které patří mezi jednu z nejrozšířenějších distribucí Linuxu. Tudíž pro ni existuje i spousta předpripravených balíčků z potřebnými aplikacemi a nástroji. A také kolem ní existuje velká komunita, kam se může člověk obrátit s případnými problémy. Druhým adeptem byla o něco méně rozšířená, avšak v poslední době velmi populární distribuce s názvem Arch Linux. Jeho hlavní výhodou jsou opravdu velmi aktuální verze programů a aplikací. Jedná se o tzv. rolling updates distribuci, to znamená, že se nevydávají jednotlivé verze distribuce, ale distribuce je aktualizovaná postupně pomocí aktualizací, kde tyto aktualizace nejsou pouze opravné verze, ale i zcela nové verze jednotlivých komponent. Co se komunity týče, tak ačkoliv není tak rozsáhlá jako u předchozího adepta, tak díky své aktivitě a odborným znalostem lidí pohybující se okolo této distribuce, svojí kvalitou komunitu okolo Ubuntu předčí. Další věcí, která mi při rozhodování mezi těmito dvěma adepty pomohla rozhodnout, byla podpora vytváření vlastních balíčků. Žádná jiná mně známá distribuce nenabízí tak jednoduchý způsob tvorby balíčků s vlastním software, jako má Arch Linux.

Nakonec jsem se tedy rozhodl pro Arch Linux. Jeho instalaci zde probírat nebudu, jelikož je to mimo rozsah této práce. Případné zájemce odkáži na wiki stránku zabívající se instalací této distribuce, která je dostupná hned v několika jazykových mutací\footnote{\url{https://wiki.archlinux.org/index.php/Official\_Installation\_Guide}}
\footnote{\url{https://wiki.archlinux.org/index.php/Official\_Installation\_Guide\_(Česky)}}
\footnote{\url{https://wiki.archlinux.org/index.php/Official\_Installation\_Guide\_(Slovenský)}}.

 

\section{Příprava virtualizačních nástrojů}
V této sekci si lehce popíšeme instalaci a konfiguraci testovaných nástrojů a dalších podpůrných utilit. Rád bych upozornil, že následující text předpokládá pokročilou znalost administrace OS Linux a alspoň částaečnou znalost distribuce Arch Linux.
\subsection{Instalace}
\subsubsection{\xen}
Při instalaci a zprovozňování Xen hypervizoru jsem narazil na pár problémů. Prvním 
\subsubsection{KVM}
\subsubsection{UML}
\subsubsection{VirtualBox}
\subsubsection{LXC}
\subsubsection{Linux-VServer}
\subsubsection{Libvirt}
\subsubsection{Virtual Machine Manager}

\subsection{Konfigurace}
\subsubsection{\xen}
\subsubsection{KVM}
\subsubsection{UML}
\subsubsection{VirtualBox}
\subsubsection{LXC}
\subsubsection{Linux-VServer}
\subsubsection{Libvirt}
\subsubsection{Virtual Machine Manager}
