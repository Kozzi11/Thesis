Je třídně založený, objektově orientovaný a víceúlohový obecný programovací jazyk, který byl vyvinut firmou Sun Microsystems. Ta jej v roce 2007 uvolnila pod svobodnou licencí GNU GPL. V dnešní době je jazyk vyvíjen zejména firmou Oracle Corporation Inc., která firmu Sun Microsystems odkoupila.

Syntaxe Javy je odvozena z jazyků C a C++, kterým je díky tomu v mnoha ohledech podobná. Ale narozdíl od těchto jazyků se zbavila konstrukcí, které často vedli k (bezpečnostním) chybám. Jedná se hlavně o konstrukce umožňující přímou práci s pamětí (ukazatele), o příkaz goto a také o beznaménkové datové typy. Díky tomu, a silné typové kontrole, jsou programy méně náchylné na bezpečnostní chyby.

Ruční správu paměti zde nahrazuje automatická správa. Tu zajišťuje garbage collector, což je proces vyhledávající (sbírající) již nedostupnou alokovanou paměť (odpad), kterou následně uvolní, aby mohla být znovu použita.

Další, včem se Java od jazyka C a C++ odlišuje, je to, že se výsledný program nekompiluje přímo na instrukční sadu dané architektury, ale do mezijazyka tzv. bytekódu (soubory s příponou \texttt{.class}). Ty jsou teprve následně "`interpretovány"' pomocí JVM (\emph{\textbf{J}ava \textbf{V}irtual \textbf{M}achine}).

Automatická správa paměti a interpretovaný běh vedl, zejména v historii, k pomalejšímu běhu výsledné aplikace. Pomalost způsobená garbage collectorem, se postupně ,s vynalézáním lepších a rychlejších algoritmů, podařila minimalizovat.

Výkonnostní problém, související s interpretací, vylepšila technika zvaná JIT (\emph{Just-in-Time}) kompilace (kompilace v době běhu). Jedná se o techniku, která dle statistiky kompiluje vhodné části kódu přímo do instrukční sady daného stroje. Tyto části kódu se při opakovaném použití znovu neinterpretují, ale jsou provedeny přímo.

Výhoda JIT kompilace se projeví nejvíce u déle běžících aplikací, kde se vykonává stejný kód dokola. Proto, a kvůli nutnosti načíst celý JVM při startu aplikace, je použití jazyka Java na malé a krátce běžící aplikce nevhodné a neefektivní.

\subsection{Reflexe a introspekce}
Jednou ze zajímavých vlastností jazyka Java je reflexe. Ta umožňuje prozkoumávat a modifikovat strukturu a chování (například proměnné, atributy, metadata a funkce) objektu přímo za běhu \ref{code:java:reflection}.
To je velmi užitečné zejména pro různe frameworky a reflexe se také hojně využíva při metaprogramování. 
\begin{figure}[!h]
\begin{lstlisting}
public class User {
    /*
     ... Some fields declaration ...
    */
    public String getName() {
        return this.name;
    }
    
    public void printName() {
        // Get name with reflection
        Class<?> c = this.class;
        System.out.println(c.getMethod("getName").invoke(this));
    }
}
\end{lstlisting}
\caption{Volání metody pomocí reflexe.}
\label{code:java:reflection}
\end{figure}

Jedná se o pokročilou vlastnost jazyka, jenž umožňuje provádět operace, které by jinak nebylo možné provádět. Umožňuje obejití zapouzdření (zpřístupňuje privátné členy) a nese tedy riziko narušení správného běhu programu. Díky tomu, a kvůli nemalé režii navíc, by měla být reflexe používána co nejméně a jen lidmi s dostatečnými znalostmi.

Introspekce využíva relexe a umožňuje získání informací o objektu, jako je například seznam všech funkcí, atributů, implementovaných rozhraní nebo název třídy \ref{code:java:introspection}.
\begin{figure}[!h]
\begin{lstlisting}
String someObject = "Some text";
// This print "String"
System.out.println(someObject.getClass().getName());
\end{lstlisting}
\caption{Výpis názvu třídy pomocí introspekce.}
\label{code:java:introspection}
\end{figure}
\subsection{Bytecode enhancement}
Podobné možnosti jako reflexe nám nabízí i bytecode enhancement. Jedná se o process, při kterém se přímo modifukuje výsledný "`bajtkód"' (obsah souboru s příponou \texttt{.class}), a který se narozdíl od reflexe neprovádí stále dokola v každém běhu, ale pouze jen jednou, a to buď po kompilaci nebo při prvním načtení třídy. Z toho vyplývá hlavní výhoda oproti reflexi, kterou je rychlost a celkově malá režie navíc. Nadruhou stranu jeho aplikování je složitější a náročnejší, než je v případě reflexe.


\subsection{Anotace}

 
