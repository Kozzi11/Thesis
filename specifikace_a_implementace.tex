V této kapitole Vám popíši základní informace o jazyce Java, ta je jedním z nejpopulárnějších OOP jazyků (podle Tiobe indexu) a taky má největší podporu ze stran různých objektových databází.

Dále Vás seznámím s některými vlastnostmi tohoto jazyka a technikami, které jednotlivé frameworky používají (například pro implementaci ORM).

Poté popíši některé hlavní specifikace a API, které se ve světě Javy používají pro řešení problému persistence objektů a persistence dat obecně. 

A nakonec představím některá existující řešení, která na těchto specifikacích staví a implementují jednotlivá API.
\section{Java}
\section{Historie}
\section{Využití}
\section{Vlastnosti}
\subsection{Reflexe}
\subsection{Introspekce}
\subsection{Anotace}

\section{Specifikace a API}
\subsection{Java Database Connectivity}
\cite{fisher:jdbc,donahue:jdpb}
\subsection{Java Data Objects}
\cite{jordan:jdo,roos:jdo,tyagi:cjdo,ezzio:uujdo}
\subsection{Java Persistence API}
\cite{jpa:spec}
\subsection{Object Data Management Group}
\cite{odmg}
\section{Existující řešní}
\subsection{Otevřené implementace}
\subsubsection{Db4objects}
\subsubsection{DataNucleus}
\subsubsection{OrientDb}
\subsubsection{Hibernate}
\subsubsection{Batoo JPA}
\subsubsection{OpenJPA}
\subsubsection{EclipseLink}
\subsubsection{TopLink Essentials}
\subsection{Komerční řešení}
\subsubsection{Objectivity/DB}
\subsubsection{ObjectDB}
\subsection{Nevyvíjené implementace}
\subsubsection{Ozone}
\subsubsection{Ojb}
\subsubsection{Speedo}
\subsubsection{TJDO}
