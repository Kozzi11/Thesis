\section{Objektové databáze}
Když se podíváme na výsledky jednotlivých testů objektových databází (grafy na obrázcích \ref{img:oodbms1} až \ref{img:oodbms13}). Zjistíme že v sedmi testech ze třinácti, tedy ve více než v polovině případů, zvýtězila databáze ObjectDB. Což odpovídá i celkovému výsledku, kde celkový čas všech testů využívajících tuto databázi, je přibližně dva a pulkrát až čtyři a pulkrát menší než u konkurenčních produktů.

Jediným testem ze zbylých šesti, kde si ObjectDB vedlo výrazněji hůře než jeho konkurence, byl test nalezení listů stromu. Důvodem je chybějící podpora některých operací v jazyce JPQL pro kolekce (v tomto případě podpora funkcí \texttt{EMPTY} a \texttt{NOT EMPTY})\footnote{Na podpoře těchto funkcí se již pracuje, a v budoucnu by se v ObjectDB měly objevit.}. Z tohoto důvodu bylo třeba kód pro nalezení listů v případě ObjectDB rozdělit na dva samostatné dotazy. Což je hlavním důvodem ztráty rychlosti.

Druhou nejrychlejší objektovou databází je dle celkového grafu (obrázek \ref{img:oodbms14}) databáze DB40. Ta byla pomalejší oproti ObjectDB hlavně v testech, kde se vytvářeli nebo aktualizovali data. Co se prohledávacích (čtecích) operací týče, zde byla DB4O většinou i mírně rychlejší než ObjectDB.

Nejpomalejším řešením se ukázala sada testů využívající NoSQL databázi OrientDB. Která nejvíce propadla v testu generování binárního stromu (graf \ref{img:oodbms9}) a v testu promazávání stromu (graf \ref{img:oodbms13}).


\section{Objektově relační mapování}
Z výsledků na grafech \ref{img:jpa1} až \ref{img:jpa14} můžeme vyčíst hned několik informací. Jednou z informací je, který relační databázový backend byl nejrychlejší. A to je HyperSQL a H2 SQL engine. Tyto dva databázový backendy podaly ve výsledku skoro shodné výsledky v závislosti na poskytovateli ORM. Velkou roli zde hraje to, že se používali v režimu "`embedded"', tedy jako ůložiště sloužil lokální soubor, ke kterému bylo přistupováno přímo přes knihovnu a né přes síťový protokol (klient--server režim), jako tomu bylo v případě MySQL a PostgreSQL.

Další informací, kterou můžeme zistit je, že v případě Hibernate a EclipseLink je MySQL backend až 2x rychlejší než PostgreSQL backend. A naopak v případě DataNucleus a OpenJPA je zase o něco málo rychlejší PostgreSQL backend než MySQL. To bych přisuzoval tomu, že EclipseLink i Hibernate jsou více vyspělé a rozšířenější implementace, které se logicky zaměřují více na optimalizace pro databázi MySQL, která je populárnější, než pro PostgreSQL.

Krom informací o výkonu jednotlivých backendech, jsou z grafů patrné i informace o jednotlivých ORM řešeních. Pokud nebudu brát v potaz PostgreSQL backend, u kterého jsou největší výkyvy, a který se nepodařilo zprovoznit v kombinaci s BatooJPA. Tak nejrychlejší JPA implementací by bylo BatooJPA. Na druhém místě by následoval Hibernate. Třetí přícku by obsadil EclipseLink. A čtvrtou pozici by obdržela implementace OpenJPA. Nejhorší výkon předvedla platforma DataNucleus.

\section{Srovnání OODBMS a ORM}
V předchozích odstavcích jsem porovnal zvlášť objektové databáze a zvlášť řešení na bázi ORM. Samozřejmě je zajímavé i porovnání mezi výkonem ORM řešení a objektovými databázemi. Ale aby toto porovnání mělo význam, je třeba vzít v úvahu, že všechny tři testované objektové databáze komunikovali s databází přímo na úrovni souborového systému. Tudíž nemé velký smysl, porovnávat jejich výkon například s ORM řešením bežícím nad MySQL databází v režimu klient server, kde je třeba brát v potaz režii spojenou s komunikačním protokolem a komunikací po síti celkově.

Proto se v následujícím textu omezím jen na porovnání objektových databází s ORM řešeními, které přistupují k datů, také přímo přes souborový systém, tedy na databáze HyperSQL a H2 SQL.

Výsledné porovnání těchto databází můžete vidět na obrázku \ref{img:resultf}.
\begin{figure}[!h]
  \includegraphics[]{obr/bench/resultf}
  \caption{Generování stromové struktury [ms]}\label{img:resultf}
\end{figure}
