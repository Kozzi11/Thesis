\section{Generování stromové struktury}
Na grafech \ref{img:oodbms1} a \ref{img:jpa1} jsou vidět výsledky jednotlivých produktů v testu generování stromové struktury, tedy rychlost jejich operací pro ukládání dat.

Při pohledu na výsledky objektových databází, můžeme konstatovat, že nejrychleji data navytváří databáze ObjectDB. Ostattní dvě databáze DB4O a OrientDB jsou skoro 5x pomalejší, než jejich konkurent.

Co se týče ORM produktů, zde je jasným výtězem BatooJPA, který je přibližně 2x až 4x rychlejší ve vytváření dat než jeho konkurenti. O druhou pozici se dělí produkty EclipseLink a OpenJPA. V těsném závěsu na třetím místě je Hibernate. Jako zcela nejpomalejší se ukázal nástroj DataNucleus a to na všech databázových ůložištích.

Pokud bychom srovnávali objektové databáze a ORM řešení, tak zde stále platí, že nejrychlejší možností jak ukládat data je použití objektové databáze ObjectDB. Ale co se ostatních objektových databází týče, tak ty jsou, v porovnání s kombinací BatooJPA a HyperSQL, až dvakrát pomalejší.
\section{Nalezení kořene}
V tomto testu (výsledky na obrázku \ref{img:oodbms2}) si z objektových databází nejlépe vedlo řešení využívající DB40. Ostatní dvě řešení byli, ale jen nepatrně pomalejší a to přibližně o 40 milisekund v případě ObjectDB a 50 milisekund v případě OrientDB.

V případě ORM řešení (obrázek \ref{img:jpa2}) je situace téměř opačná jako u předchozího testu. Tentokrát je nejrychlejším produktem DataNucleus následovaný produktem Hibernate a BatooJPA. A nejpomalejší nalezení kořene stromu se dosáhlo řešením využívající EclipseLink.

Rychlosti objektových databází a objektově--relačního mapování jsou v tomto testu velmi podobné, ale celkově jsou řešení využívající mapovaní mírně rychlejší. Teda až na EclipseLink, který je celkově nejpomalejším řešením.  

\section{Aktualizování kořene}

\section{Nalezení listů}
\section{Přidání listů}
\section{Nalezení uzlů (procházením)}
\section{Nalezení uzlů (dotazem)}
\section{Průchod stromem}
\section{Binární strom}
\section{Prohození potomků kořene}
\section{Průchod binárním stromem}
\section{Výpočet výšky stromu}
\section{Promazání stromu}
\section{Celkový čas všech testů}
