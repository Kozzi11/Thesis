Všechny nástroje prezentovány v této práci mají určitě svoje využití a nedá se určit jasný vítěz. Velmi totiž záleží právě na účelu pro jaký má být virtualizace použita, co se od ní očekává a co všechno musí umět a podporovat.

Pokud bych měl zhodnotit čistě výkon jednotlivých virtualizačních nástrojů, založený na výsledcích mého testování, tak nejlepšího výkonu bylo dosaženo při použití kontejnerových řešení. A to zejména s distribucí Fedora Core 16 jako hostem. Průměrná režie těchto řešení je pod 5\,\%. Na druhou stranu LXC i Linux-VServer mají svá značná omezení, která plynou z faktu že se nejedná a plnohodnotnou virtualizaci. Tudíž ačkoliv nabízí perfektní výkon a využití zdrojů, nebudou vždy možné použít. Zejména ne tam, kde je primárním cílem izolace a stabilita systému.

V takovém prostředí se spíš vyplatí nasadit některé z plnohodnotných virtualizačních řešení (KVM, Xen nebo VirtualBox). Pokud se podíváme na mé testy, tak zjistíme, že například VirtualBox co se týče výkonu za svými konkurenty pokulhává. Na druhou stranu pokud nám nevadí určitá režie navíc, nabízí nám velmi dobře propracované ovládací rozhraní. Což může být pro hodně lidí velmi důležité. Například v pracovním prostředí, kde potřebuji testovat software na nejrůznějších systémech a konfiguracích, by pro mě snadnost vytvoření a nastavení virtuálního stroje hrála větší roli, než právě výkon.

Na druhou stranu v prostředí, kde bych měl virtualizovaný například webový a databázový server, bych vyžadoval pokud možno co nejlepší výkon. Potom bych měl na výběr buď Xen hypervizor nebo KVM. V testech ve většině případů dosáhlo lepšího skóre řešení využívající KVM. Díky tomu že je součástí jádra je sním i mnohem méně práce při instalaci a konfiguraci. Proto bych jej upřednostnil před Xen hypervizorem.

Na druhou stranu Xen se dá použít i v prostředí, kde hostitel nepodporuje HW virtualizaci. Zde by volba padla určitě právě na Xen, tedy pokud by daný systém byl dostupný v modifikované verzi pro běh pod Xenem. V opačném případě by řešením bylo použít VirtualBox.

V této práci jsem popsal jednotlivé nástroje s jejich kladnými i zápornými vlastnostmi. V jednotlivých testech jsem se snažil zachytit jejich výkonnost a tím poukázat, který nástroj se jak hodí pro určitou činnost. Tato práce by se samozřejmě dala o několik věcí rozšířit.

Například by mohlo být zajímavé provést ty samé testy na odlišném hostiteli. V ideálním případě odlišném nejen například distribucí systému Linux, ale také s rozdílnou HW konfigurací. Ta by mohla obsahovat procesor od firmy Intel, což by nám mohlo dát představu, která z firem má technologii HW podpory virtualizace lépe zvládnutou. Nebo také to, která je v dnešní době lépe implementovaná v současných verzích virtualizačních nástrojů.

Dále by stálo za to, otestovat různé hodnoty konfigurace jádra, jako například hodnota časovače přerušení (toto jsem popravdě zkoušel u Apache benchmark testu na sestavě KVM a Arch Linux, výkon byl zhruba o 10\% lepší). Například u distribuce Fedora se dá časovač nastavit parametrem jádra při spouštění. Což je výhoda, jelikož není potřeba celé jádro znova zkompilovat. I další změny by mohli mít vliv na výkon. Například zkompilování jádra s podporou přímo konkrétní sady instrukcí pro daný procesor, či zkusit místo standardního I/O plánovače CFQ použít plánovač Deadline). Je toho určitě i mnohem víc, ale to by vydalo na celou knihu a pokud bych to měl vše zahrnout do této práce, nevešel bych se do očekávaného rozsahu této práce.

I tak doufám, že pro případného čtenáře byla tato práce přínosem a pomohla mu s rozšířením jeho znalostí o virtualizačních nástrojích nebo si zvolit vhodný virtualizační nástroj pro jeho potřebu. \cite{dike:uml}
\cite{hagen:xen} \cite{huynh:kvm} \cite{kivity:kvm} \cite{romero:vbox} \cite{ruest:virt} \cite{wiki:ovz}
