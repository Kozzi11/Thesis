Mým cílem v této práci bylo seznámit čtenáře s problematikou persistence objektů a jejich stavů v OOP jazycích, zejména v programovacím jazyce Java. Ale i přes zaměření na tento programovací jazyk, jsou informace uvedené v této práci platné z velké části i pro ostatní OOP jazyky. Toto platí zejména pro první dvě teoretické kapitoly.

Ve zbylých kapitolách jsem se více zaměřil na popis jazyka Javy a jednotlivých nástrojů a specifikací, jenž jsou v tomto jazyce využívány pro persistenci dat.

Během psaní této práce, jsem si rozšířil svoje znalosti o hodně zajímavých informací. A během implementace podpůrného programu, jenž jsem využil pro testování výkonu jednotlivých nástrojů, jsem narazil na několik zajímavých problémů. A to zejména na fakt, že specifikace je jedna věc, ale implementace je věc druhá.

Ačkoliv všechny mnou testované ORM nástroje by měli splňovat JPA specifikaci, jen málo kdy se podařilo napsat kód využívající JPA API tak, aby běžel ve všech mnou testovaných implementacích.

Asi nejméně problematickými nástroji byly v tomto směru Hibernate a EclipseLink. Největší potíže jsem měl s DataNucleus, u kterého se mi jeden test nepodařilo implementovat vůbec.

Docela zábavná byla i situace, kdy jedna implementace (BatooJPA) měla prohozený význam funkcí \texttt{EMPTY} a \texttt{NOT EMPTY}. Což na můj popud bylo vyvojáři rychle opraveno, ale bohužel jen částečně. Sice opravili toto chování v kódu, ale při snaze skompilovat opravenou verzi, to zhavarovalo na jednotkových testech, kde se stále počítalo se špatným chováním těchto funkcí.

Naštěstí se jedná o otevřený software, jenž má svůj kód dostupný na githubu, takže nebyl problém si vytvořit fork repositáře a opravit rozbité testy. Na můj požadavek začlenit tuto opravu bylo zareagováno opravdu rychle. To hodnotím velmi pozitivně. Dokonce jsem během testování narazil i na pár dalších chyb, které jsem touto cestou odstranil. Takže bylo nakonec možné tuto implementaci zařadit do mého testování, teda až na vyjímku PostgreSQL databáze, s touto databází se mi testy nepodařilo provést a na nalezení a opravení chyby mi již nezbyl čas.

Další zajímavý problém jsem měl s OpenJPA implementací. Zde se sice nejednalo o problém nesouladu specifikace a implementace, ale o výkonnostní potíže při určitém typu operací. To některé mé testy degradovalo natolik, že jejich provedení trvalo místo obvyklích pár vteřin i několik desítek minut. Naštěstí pár dní před odevzdáním této práce vyšla aktualizovaná verze této implementace. Kde se již naštěstí tento problém nevyskytoval, takže výsledky uvedené v páté kapitole nejsou tímto nijak poznamenány.

Porovnání jednotlivých nástrojů bylo zaměřeno na jejich výkon, tedy na rychlost jednotlivých operací. Což je jeden z nejzákladnějších a nejzajímavějších faktorů při volbě daného nástroje. Na druhou stranu zde existuje prostor pro rozšíření práce o měření dalších faktorů. A to zejména měření nárkoů na systémové prostředky jako je procesor, operační paměť nebo velikost zabraného místa na disku.

Jelikož veškeré testy jsem prováděl s výchozím nastavením jednotlivých nástrojů. Je zde i možnost zaměřit se na optimalizaci nastavení jednotlivých nástrojů a databází. Ale rozsah tohoto téma by vydal na samostatnou práci. 

%Ve čtvrté kapitole jsem se snažil čtenáře seznámit s programovacím jazykem Java a popsat mu %některé techniky a vlastnosti jazyka, jenž jsou používané různými nástroji pro podporu persistence objektů. 

%Dále jsem v téže kapitole popsal existující nástroje a specifikace, které ve světě jazyka Java existují a řeší problémy persistence dat či objketů, objektově relační mapování, transakční zpracování a mnoho dalších souvisejících ůloh.

%V posledních dvou kapitolách jsem se zaměřil na výkonnostní testování některých existujících nástrojů pro persistenci objektů. Tyto nástroje jsem rozdělil do dvou skupin. Kde první skupinou byly objektové databáze a druhou skupinou byly nástroje mapující objekty na relační databáze.

%Tyto jednotlivé nástroje a databáze jsem otestoval pomocí mnou napsané aplikace, která testovala výkonost jednotlivých řešení pomocí sady testů. Výsledky těchto testů jsem následně 
