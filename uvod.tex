Objektové paradigma a rozšíření objektově orientovaných programovacích jazyků, změnilo zcela pohled na návrh a psaní programů. Narozdíl od procedurálního paradigma, do té doby nejrozšířenější paradigma, se zde místo s pojmy data, procedury a funkce, pracuje s pojmy jako je třída, objekt (a jeho stav), metoda, zpráva, dedičnost a podobně.

Tato změna pohledu programátora na návrh aplikace, přinesla i nové požadavky co se persistence aplikace týče. Do té doby zde existovala potřeba někde ukládat data, což bylo většinou zajištěno pomocí relační databáze, která se pro tyto účely perfektně hodila a poskytovala i další výhody. Například logické dělení dat do tabulek nebo efektivní vyhledávání nad daty pomocí indexů atd. Tato data byla většinou primitivní (čísla, řetězce) a dala se tedy snadno mapovat na typy podporované v dané databázi.

U objektově orientovaného programování máme místo jednoduchých dat objekty. Což jsou složité datové struktury, které mohou obsahovat další objekty ale i primitivní data. Proto zde vyvstala potřeba nástrojů umožňujících persistenci celých objektů.

Podrobnější informace o tomto problému, jak se dá řešit, jaké existují již hotové řešení a jaký je jejich výkon se Vám pokusím popsat v následujících kapitolách.
