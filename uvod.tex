Virtualizace v dnešní době zažívá velký rozkvět. Z mainstreamového segmentu se rozšířila i do oblastí středních a malých firem a v poslední době se sní čím dál tím více můžeme setkat i na domácích počítačích. Hlavní úděl na tom nese rozvoj virtualizace x86 architektury a také hardwarová podpora virtualizace, kterou v dnešní době má většina procesorů jak v obyčejných domácích PC sestavách, tak i v noteboocích.

S virtualizace se tak stal velký byznys a na trhu tak existuje nemálo komerčních řešení od různých firem (pro zájemce o podrobnější informace o některých těchto produktech odkáži na knihu \cite{hess:pract}). Tyto firmy nabízí různé varianty jak pro obyčejné domácí využití, tak i pro malé, střední či velké firmy. Ne úplně každý chce však za virtualizaci platit. A proto vzniklo spoustu neplacených produktů. Některé z nich jsou jen omezenou verzí svých komerčních bratříčků a jsou spíš takovým lákadlem pro firmy, aby si na ně zvykli a až jim jejich funkcionalita nebude dostačovat, tak nejspíše sáhnou po některé placené verzi od stejného výrobce. Já se v této práci budu zaměřovat na produkty, které nejsou jen zdarma ale jsou i open source, to znamená, že jsou volně dostupné jejich zdrojové kódy. U jednotlivých vybraných produktů vás seznámím s jejich vlastnostmi, ukáži jejich instalaci a konfiguraci a nakonec provedu několik výkonnostních testů těchto řešení.
