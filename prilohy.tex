\chapter{\xen příprava balíčku}
\section{PKGBUILD}

\begin{lstlisting}
pkgname=xen-rc
pkgver=23276
pkgrel=1
pkgdesc="Xen 4.1.3 rc (hypervisor tools and doc) HG"
arch=(i686 x86_64)
url="http://xen.org/"
license="GPL"
depends=('bzip2' 'iproute' 'bridge-utils' 'python2' 'sdl'
         'zlib' 'e2fsprogs' 'bin86' 'pkgconfig' 'gnutls'
         'lzo2' 'glibc')
makedepends=('dev86' 'mercurial' 'git' 'ghostscript')
conflicts=('xen' 'xen3' 'xen4' 'xen-hv-tools' 'libxen4')
provides=('xen')
source=(archinit.patch texi2html.patch 09_xen)

md5sums=('d3ab9bbae472e613a04dc8e62377ed93'
         'c94602f1feaa5d968db1e9f640dfd2a5'
         '3a3240e1edde3a8e295f928be52dbde4')
			   
_hgroot="http://xenbits.xensource.com/"
_hgrepo="xen-4.1-testing.hg"

build() {

  cd "$srcdir"
  msg "Connecting to Mercurial server...."

  if [ -d $_hgrepo ] ; then
    cd $_hgrepo
    hg pull -u || return 1
    msg "The local files are updated."
  else
    hg clone $_hgroot $_hgrepo || return 1
  fi

  msg "Mercurial checkout done or server timeout"
  msg "Starting make..."

  rm -rf "$srcdir/$_hgrepo-build"
  cp -r "$srcdir/$_hgrepo" "$srcdir/$_hgrepo-build"
  cd "$srcdir/$_hgrepo-build"

  patch -p1 -F99 -i ../archinit.patch
  patch -p1 -i ../texi2html.patch
  unset CFLAGS LDFLAGS

  make PYTHON=python2 DESTDIR=$pkgdir  xen
  make PYTHON=python2 DESTDIR=$pkgdir  tools  
    
}

package() {

  cd "$srcdir/$_hgrepo-build"
  unset CFLAGS LDFLAGS
  make PYTHON=python2 DESTDIR=$pkgdir  install-xen
  make PYTHON=python2 DESTDIR=$pkgdir  install-tools  
  
  sed -i 's#XENDOM_CONFIG=/etc/sysconfig/xendomains#XENDOM_CONFIG=/etc/conf.d/xendomains#' $pkgdir/etc/init.d/xendomains
  sed -i "s#touch /var/lock/subsys/xend#mkdir -p /var/lock/subsys\n     touch /var/lock/subsys/xend#" $pkgdir/etc/init.d/xend

  [ -d $pkgdir/usr/lib64 ] && ( cd $pkgdir/usr && cp -R lib64/* lib/ && rm -R lib64 )
  ( cd $pkgdir/etc && mv init.d rc.d ) || return 1
  rm -f $pkgdir/usr/share/man/man1/qemu-img.1* \
       $pkgdir/usr/share/man/man1/qemu.1*
  # First experiment to generate grub2.cfg entry
  mkdir -p $pkgdir/etc/grub.d
  chmod +x $srcdir/09_xen
  cp $srcdir/09_xen  $pkgdir/etc/grub.d

  ############ kill unwanted stuff ############
  # stubdom: newlib
  rm -rf $pkgdir/usr/*-xen-elf

  # hypervisor symlinks
  rm -rf $pkgdir/boot/xen-4.1.gz
  rm -rf $pkgdir/boot/xen-4.gz
  rm -rf $pkgdir/boot/xen.gz

  # silly doc dir fun
  rm -fr $pkgdir/usr/share/doc/xen
  rm -rf $pkgdir/usr/share/doc/qemu

  # Pointless helper
  rm -f $pkgdir/usr/sbin/xen-python-path

  # qemu stuff (unused or available from upstream)
  rm -rf $pkgdir/usr/share/xen/man
  rm -rf $pkgdir/usr/bin/qemu-*-xen
  for file in bios.bin openbios-sparc32 openbios-sparc64 ppc_rom.bin \
         pxe-e1000.bin pxe-ne2k_pci.bin pxe-pcnet.bin pxe-rtl8139.bin \
         vgabios.bin vgabios-cirrus.bin video.x openbios-ppc bamboo.dtb
  do
        rm -f $pkgdir/usr/share/xen/qemu/$file
  done

  # adhere to Static Library Packaging Guidelines
  rm -rf $pkgdir/usr/lib/*.a 	
}
\end{lstlisting}

\section{archinit.patch}
\begin{lstlisting}
diff -Naur orig.xen-4.1.1//tools/hotplug/Linux/init.d/xencommons xen-4.1.1//tools/hotplug/Linux/init.d/xencommons
--- orig.xen-4.1.1//tools/hotplug/Linux/init.d/xencommons	2011-07-03 03:08:44.953747064 -0700
+++ xen-4.1.1//tools/hotplug/Linux/init.d/xencommons	2011-07-05 13:47:54.627029164 -0700
@@ -18,6 +18,9 @@
 # Description:       Starts and stops the daemons neeeded for xl/xend
 ### END INIT INFO
 
+. /etc/rc.conf
+. /etc/rc.d/functions
+
 if [ -d /etc/sysconfig ]; then
 	xencommons_config=/etc/sysconfig
 else
@@ -26,7 +29,7 @@
 
 test -f $xencommons_config/xencommons && . $xencommons_config/xencommons
 
-XENCONSOLED_PIDFILE=/var/run/xenconsoled.pid
+XENCONSOLED_PIDFILE=/run/daemons/xenconsoled.pid
 shopt -s extglob
 
 if test "x$1" = xstart && \
@@ -60,33 +64,39 @@
 		    time=$(($time+1))
                     sleep 1
                 done
-		echo
-
 		# Exit if we timed out
 		if ! [ $time -lt $timeout ] ; then
-		    echo Could not start xenstored
+		    #echo Could not start xenstored
+                    stat_fail
 		    exit 1
 		fi
+                stat_done
 
-		echo Setting domain 0 name...
+		stat_busy "Setting domain 0 name..."
 		xenstore-write "/local/domain/0/name" "Domain-0"
+		stat_done
 	fi
 
-	echo Starting xenconsoled...
+	#echo Starting xenconsoled...
+	stat_busy "Starting xenconsoled"
 	test -z "$XENCONSOLED_TRACE" || XENCONSOLED_ARGS=" --log=$XENCONSOLED_TRACE"
 	xenconsoled --pid-file=$XENCONSOLED_PIDFILE $XENCONSOLED_ARGS
 	test -z "$XENBACKENDD_DEBUG" || XENBACKENDD_ARGS="-d"
 	test "`uname`" != "NetBSD" || xenbackendd $XENBACKENDD_ARGS
+	stat_done
+	add_daemon xencommons
 }
 do_stop () {
-        echo Stopping xenconsoled
+         stat_busy "Stopping xenconsoled"
 	if read 2>/dev/null <$XENCONSOLED_PIDFILE pid; then
 		kill $pid
 		while kill -9 $pid >/dev/null 2>&1; do sleep 0.1; done
 		rm -f $XENCONSOLED_PIDFILE
 	fi
+	stat_done
 
-	echo WARNING: Not stopping xenstored, as it cannot be restarted.
+	printhl "WARNING: Not stopping xenstored, as it cannot be restarted."
+        rm_daemon xencommons
 }
 
 case "$1" in
diff -Naur orig.xen-4.1.1//tools/hotplug/Linux/init.d/xend xen-4.1.1//tools/hotplug/Linux/init.d/xend
--- orig.xen-4.1.1//tools/hotplug/Linux/init.d/xend	2011-07-03 03:08:44.953747064 -0700
+++ xen-4.1.1//tools/hotplug/Linux/init.d/xend	2011-07-05 01:47:40.981951191 -0700
@@ -18,6 +18,10 @@
 # Description:       Starts and stops the Xen control daemon.
 ### END INIT INFO
 
+. /etc/rc.conf
+. /etc/rc.d/functions
+
+
 shopt -s extglob
 
 # Wait for Xend to be up
@@ -37,23 +41,30 @@
 case "$1" in
   start)
 	if [ -z "`ps -C xenconsoled -o pid=`" ]; then
-		echo "xencommons should be started first."
+	 printhl "xencommons should be started first."
 		exit 1
 	fi
 	# mkdir shouldn't be needed as most distros have this already created. Default to using subsys.
 	# See docs/misc/distro_mapping.txt
-	mkdir -p /var/lock
-	if [ -d /var/lock/subsys ] ; then
-		touch /var/lock/subsys/xend
+	if [ -d /run/lock/subsys ] ; then
+		touch /run/lock/subsys/xend
 	else
-		touch /var/lock/xend
+		touch /run/lock/xend
 	fi
+	stat_busy "Starting xend"
 	xend start
 	await_daemons_up
+	stat_done
+	add_daemon xend
 	;;
+
+
   stop)
+   stat_busy "Stopping xend"
 	xend stop
-	rm -f /var/lock/subsys/xend /var/lock/xend
+	rm -f /run/lock/xend /var/lock/xend
+	stat_done
+	rm_daemon xend
 	;;
   status)
 	xend status
@@ -62,8 +73,10 @@
         xend reload
         ;;
   restart|force-reload)
+   stat_busy "Restarting xend"
 	xend restart
 	await_daemons_up
+	stat_done
 	;;
   *)
 	# do not advertise unreasonable commands that there is no reason
diff -Naur orig.xen-4.1.1//tools/hotplug/Linux/init.d/xendomains xen-4.1.1//tools/hotplug/Linux/init.d/xendomains
--- orig.xen-4.1.1//tools/hotplug/Linux/init.d/xendomains	2011-07-03 03:08:44.953747064 -0700
+++ xen-4.1.1//tools/hotplug/Linux/init.d/xendomains	2011-07-05 13:46:36.208222760 -0700
@@ -26,6 +26,9 @@
 # Description:       Start / stop domains automatically when domain 0 
 #                    boots / shuts down.
 ### END INIT INFO
+. /etc/rc.conf
+. /etc/rc.d/functions
+
 
 CMD=xm
 $CMD list &> /dev/null
@@ -46,93 +49,52 @@
 	exit 0
 fi
 
-# See docs/misc/distro_mapping.txt
-if [ -d /var/lock/subsys ]; then
-	LOCKFILE=/var/lock/subsys/xendomains
-else
-	LOCKFILE=/var/lock/xendomains
-fi
-
-if [ -d /etc/sysconfig ]; then
-	XENDOM_CONFIG=/etc/sysconfig/xendomains
-else
-	XENDOM_CONFIG=/etc/default/xendomains
-fi
+LOCKFILE=/run/lock/xendomains
+XENDOM_CONFIG=/etc/default/xendomains
 
-test -r $XENDOM_CONFIG || { echo "$XENDOM_CONFIG not existing";
+test -r $XENDOM_CONFIG || { 
+	printhl "$XENDOM_CONFIG not existing";
 	if [ "$1" = "stop" ]; then exit 0;
 	else exit 6; fi; }
 
 . $XENDOM_CONFIG
 
-# Use the SUSE rc_ init script functions;
-# emulate them on LSB, RH and other systems
-if test -e /etc/rc.status; then
-    # SUSE rc script library
-    . /etc/rc.status
-else    
-    _cmd=$1
-    declare -a _SMSG
-    if test "${_cmd}" = "status"; then
+_cmd=$1
+declare -a _SMSG
+if test "${_cmd}" = "status"; then
 	_SMSG=(running dead dead unused unknown)
 	_RC_UNUSED=3
-    else
+else
 	_SMSG=(done failed failed missed failed skipped unused failed failed)
 	_RC_UNUSED=6
-    fi
-    if test -e /etc/init.d/functions; then
-	# REDHAT
-	. /etc/init.d/functions
-	echo_rc()
-	{
-	    #echo -n "  [${_SMSG[${_RC_RV}]}] "
-	    if test ${_RC_RV} = 0; then
-		success "  [${_SMSG[${_RC_RV}]}] "
-	    else
-		failure "  [${_SMSG[${_RC_RV}]}] "
-	    fi
-	}
-    elif test -e /lib/lsb/init-functions; then
-	# LSB    
-    	. /lib/lsb/init-functions
-        if alias log_success_msg >/dev/null 2>/dev/null; then
-	  echo_rc()
-	  {
-	       echo "  [${_SMSG[${_RC_RV}]}] "
-	  }
-        else
-	  echo_rc()
-	  {
-	    if test ${_RC_RV} = 0; then
-		log_success_msg "  [${_SMSG[${_RC_RV}]}] "
-	    else
-		log_failure_msg "  [${_SMSG[${_RC_RV}]}] "
-	    fi
-	  }
-        fi
-    else    
-	# emulate it
-	echo_rc()
-	{
-	    echo "  [${_SMSG[${_RC_RV}]}] "
-	}
-    fi
-    rc_reset() { _RC_RV=0; }
-    rc_failed()
-    {
+fi
+
+
+
+echo_rc() {
+	echo
+	printhl "Return Status: ${_SMSG[${_RC_RV}]}"
+}
+
+
+rc_reset() { _RC_RV=0; }
+
+
+rc_failed() {
 	if test -z "$1"; then 
-	    _RC_RV=1;
+		_RC_RV=1;
 	elif test "$1" != "0"; then 
-	    _RC_RV=$1; 
-    	fi
+		_RC_RV=$1; 
+	fi
 	return ${_RC_RV}
-    }
-    rc_check()
-    {
+}
+
+rc_check() {
 	return rc_failed $?
-    }	
-    rc_status()
-    {
+}	
+
+
+rc_status() {
 	rc_failed $?
 	if test "$1" = "-r"; then _RC_RV=0; shift; fi
 	if test "$1" = "-s"; then rc_failed 5; echo_rc; rc_failed 3; shift; fi
@@ -140,26 +102,24 @@
 	if test "$1" = "-v"; then echo_rc; shift; fi
 	if test "$1" = "-r"; then _RC_RV=0; shift; fi
 	return ${_RC_RV}
-    }
-    rc_exit() { exit ${_RC_RV}; }
-    rc_active() 
-    {
+}
+
+
+rc_exit() { exit ${_RC_RV}; }
+
+
+rc_active() {
 	if test -z "$RUNLEVEL"; then read RUNLEVEL REST < <(/sbin/runlevel); fi
 	if test -e /etc/init.d/S[0-9][0-9]${1}; then return 0; fi
 	return 1
-    }
-fi
+}
 
-if ! which usleep >&/dev/null
-then
-  usleep()
-  {
-    if [ -n "$1" ]
-    then
-      sleep $(( $1 / 1000000 ))
-    fi
-  }
-fi
+usleep() {
+	if [ -n "$1" ]
+	then
+	  sleep $(( $1 / 1000000 ))
+	fi
+}
 
 # Reset status of this service
 rc_reset
@@ -235,10 +195,12 @@
 start() 
 {
     if [ -f $LOCKFILE ]; then 
-	echo -e "xendomains already running (lockfile exists)"
+	stat_busy "xendomains already running (lockfile exists)"
+	stat_fail
 	return; 
     fi
 
+    printhl "Starting Xen Domains"
     saved_domains=" "
     if [ "$XENDOMAINS_RESTORE" = "true" ] &&
        contains_something "$XENDOMAINS_SAVE"
@@ -299,6 +261,7 @@
 	    fi
 	done
     fi
+    add_daemon xendomains
 }
 
 all_zombies()
@@ -352,7 +315,7 @@
     if test "$XENDOMAINS_AUTO_ONLY" = "true"; then
 	rdnames
     fi
-    echo -n "Shutting down Xen domains:"
+    printhl "Shutting down Xen domains"
     name=;id=
     while read LN; do
 	parseln "$LN" || continue
@@ -465,6 +428,7 @@
     rm -f $LOCKFILE
     
     exec 2>&3
+    rm_daemon xendomains
 }
 
 check_domain_up()
diff -Naur orig.xen-4.1.1//tools/hotplug/Linux/init.d/xen-watchdog xen-4.1.1//tools/hotplug/Linux/init.d/xen-watchdog
--- orig.xen-4.1.1//tools/hotplug/Linux/init.d/xen-watchdog	2011-07-03 03:08:44.957080397 -0700
+++ xen-4.1.1//tools/hotplug/Linux/init.d/xen-watchdog	2011-07-05 13:20:22.515289867 -0700
@@ -17,49 +17,32 @@
 ### END INIT INFO
 #
 
+. /etc/rc.conf
+. /etc/rc.d/functions
+
 DAEMON=/usr/sbin/xenwatchdogd
 base=$(basename $DAEMON)
+initname="xen-watchdog"
 
-# Source function library.
-if [ -e  /etc/init.d/functions ] ; then
-    . /etc/init.d/functions
-elif [ -e /lib/lsb/init-functions ] ; then
-    . /lib/lsb/init-functions
-    success () {
-        log_success_msg $*
-    }
-    failure () {
-        log_failure_msg $*
-    }
-else
-    success () {
-        echo $*
-    }
-    failure () {
-        echo $*
-    }
-fi
 
 start() {
 	local r
-	echo -n $"Starting domain watchdog daemon: "
+	stat_busy "Starting domain watchdog daemon"
 
 	$DAEMON 30 15
 	r=$?
-	[ "$r" -eq 0 ] && success $"$base startup" || failure $"$base startup"
-	echo
+	[ "$r" -eq 0 ] && stat_done ; add_daemon $initname || stat_fail
 
 	return $r
 }
 
 stop() {
 	local r
-	echo -n $"Stopping domain watchdog daemon: "
+	stat_busy "Stopping domain watchdog daemon"
 
 	killall -USR1 $base 2>/dev/null
 	r=$?
-	[ "$r" -eq 0 ] && success $"$base stop" || failure $"$base stop"
-	echo
+	[ "$r" -eq 0 ] && stat_done ; rm_daemon $initname || stat_fail
 
 	return $r
 }

\end{lstlisting}

\section{texi2html.patch}
\begin{lstlisting}
diff -Naur xen-4.1-testing.hg.orig/tools/Makefile xen-4.1-testing.hg/tools/Makefile
--- xen-4.1-testing.hg.orig/tools/Makefile	2012-03-18 09:32:20.974961585 +0100
+++ xen-4.1-testing.hg/tools/Makefile	2012-03-18 09:07:37.000000000 +0100
@@ -107,6 +107,7 @@
 	set -e; \
 		$(buildmakevars2shellvars); \
 		cd ioemu-dir; \
+		sed -i 's/number[ ]/number-sections /' Makefile; \
 		$(QEMU_ROOT)/xen-setup $(IOEMU_CONFIGURE_CROSS)
 
 .PHONY: ioemu-dir-force-update
\end{lstlisting}

\section{09\_xen}
\begin{lstlisting}
#! /bin/sh -e

if [ -f /usr/lib/grub/grub-mkconfig_lib ]; then
  . /usr/lib/grub/grub-mkconfig_lib
else
  # no grub file, so we notify and exit gracefully
  echo "Cannot find grub config file, exiting." >&2
  exit 0
fi

XEN_HYPERVISOR_CMDLINE="xsave=1"
XEN_LINUX_CMDLINE="console=tty0"
[ -r /etc/xen/grub.conf ] && . /etc/xen/grub.conf

CLASS="--class gnu-linux --class gnu --class os"

if [ "x${GRUB_DISTRIBUTOR}" = "x" ] ; then
  OS=GNU/Linux
else
  OS="${GRUB_DISTRIBUTOR} GNU/Linux"
  CLASS="--class $(echo ${GRUB_DISTRIBUTOR} | tr '[A-Z]' '[a-z]' | cut -d' ' -f1) ${CLASS}"
fi

# loop-AES arranges things so that /dev/loop/X can be our root device, but
# the initrds that Linux uses don't like that.
case ${GRUB_DEVICE} in
  /dev/loop/*|/dev/loop[0-9])
    GRUB_DEVICE=`losetup ${GRUB_DEVICE} | sed -e "s/^[^(]*(\([^)]\+\)).*/\1/"`
  ;;
esac

if [ "x${GRUB_DEVICE_UUID}" = "x" ] || [ "x${GRUB_DISABLE_LINUX_UUID}" = "xtrue" ] \
    || ! test -e "/dev/disk/by-uuid/${GRUB_DEVICE_UUID}" \
    || [ "`grub-probe -t abstraction --device ${GRUB_DEVICE} | sed -e 's,.*\(lvm\).*,\1,'`" = "lvm"  ] ; then
  LINUX_ROOT_DEVICE=${GRUB_DEVICE}
else
  LINUX_ROOT_DEVICE=UUID=${GRUB_DEVICE_UUID}
fi

xen_entry ()
{
  os="$1"
  xen_version="$2"
  version="$3"
  xen_args="$4"
  args="$5"
  printf "menuentry 'Xen %s / %s, with Linux %s' --class xen ${CLASS} {\n" "${xen_version}" "${os}" "${version}"
  save_default_entry | sed -e "s/^/\t/"

  if [ -z "${prepare_boot_cache}" ]; then
    prepare_boot_cache="$(prepare_grub_to_access_device ${GRUB_DEVICE_BOOT} | sed -e "s/^/\t/")"
  fi
  printf '%s\n' "${prepare_boot_cache}"
  cat << EOF
       echo    '$(printf "Loading Xen %s ..." ${xen_version})'
       multiboot       ${rel_dirname}/${xen_basename} ${rel_dirname}/${xen_basename} ${xen_args}
       echo    $(printf "$(gettext "Loading Linux %s ...")" ${version})
       module  ${rel_dirname}/${basename} ${rel_dirname}/${basename} root=${linux_root_device_thisversion} ro ${args}
EOF
  if test -n "${initrd}" ; then
    cat << EOF
       echo    "Loading initial ramdisk ..."
       module  ${rel_dirname}/${initrd}
EOF
  fi
  cat << EOF
}
EOF
}

xen_list=`for i in /boot/xen-*.gz /xen-*.gz ; do
       if grub_file_is_not_garbage "$i" ; then echo -n "$i "; fi
done`
prepare_boot_cache=

while [ "x$xen_list" != "x" ] ; do
  xen=`version_find_latest $xen_list`
  echo "Found Xen hypervisor image: $xen" >&2
  xen_basename=`basename $xen`
  xen_dirname=`dirname $xen`
  rel_xen_dirname=`make_system_path_relative_to_its_root $xen_dirname`
  xen_version=`echo $xen_basename | sed -e "s,^[^0-9]*-,,g" | sed -e "s,.gz,,g"`
  alt_xen_version=`echo $xen_version | sed -e "s,\.old$,,g"`

  list="/boot/vmlinuz-linux";

  while [ "x$list" != "x" ] ; do
    linux=`version_find_latest $list`
    echo -e "\tFound linux image: $linux" >&2
    basename=`basename $linux`
    dirname=`dirname $linux`
    rel_dirname=`make_system_path_relative_to_its_root $dirname`
    version=`echo $basename | sed -e "s,^[^0-9]*-,,g"`
    base_init=`echo $basename | sed -e "s,vmlinuz,initramfs,g"`
    alt_version="${base_init}-fallback"
    linux_root_device_thisversion="${LINUX_ROOT_DEVICE}"
    initrd=

    for i in "${base_init}.img"; do
       if test -e "${dirname}/${i}" ; then
         initrd="$i"
         break
       fi
    done
    if test -n "${initrd}" ; then
      echo -e "\tFound initrd image: ${dirname}/${initrd}" >&2
    else
      # "UUID=" magic is parsed by initrds.  Since there's no initrd, it can't work here.
      linux_root_device_thisversion=${GRUB_DEVICE}
    fi

    xen_entry "${OS}" "${xen_version}" "${version}" \
        "${XEN_HYPERVISOR_CMDLINE}" \
       "${XEN_LINUX_CMDLINE}"

    list=`echo $list | tr ' ' '\n' | grep -vx $linux | tr '\n' ' '`
  done

  xen_list=`echo $xen_list | tr ' ' '\n' | grep -vx $xen | tr '\n' ' '`
done
\end{lstlisting}

%\chapter{Obsah CD}
%\chapter{Manual}
%\chapter{Konfigrační soubor}
%\chapter{RelaxNG Schéma konfiguračního soboru}
%\chapter{Plakat}
