V předchozí části jsem stručně představil jazyk Java a některé jeho vlastnosti, které nějak souvisí s možností ukládání objektů jazyka Java do databáze. V této části se zaměřím na specifikace, jenž popisují způsob, jak ukládat data do databáze a definují API, které se pro tyto účely má používat. 

\subsection{JDBC\texttrademark}
JDBC někdy též označováno jako \textit{Java Database Connectivity} je základní aplikační rozhraní (API) jazyka Java pro přístup k relačním databázím. Přesněji pro přístup k jakýmkoliv tabulkovým datům, jelikož JDBC podporuje například i soubory obsahující tabulková data \cite{fisher:jdbc,donahue:jdpb}.

JDBC se skládá z množství tříd a rozhraní umožňující jednotý přístup k relačním datům a provádění (SQL) dotazů nad nimy. Základní tři funkce, které JDBC nabízí jsou:
\begin{itemize}
  \item Připojení k databázi nebo libovolném zdroji tabulkových dat.
  \item Poslání SQL dotazu na zdroj dat.
  \item Zpracování výsledku dotazu.
\end{itemize}

Jednoduchý příklad těchto tří funkcí za použití JDBC API jsem znázornil na ukázce kódu \ref{jdbc:conn}.
\begin{figure}
\begin{lstlisting}
// Connect to datbase with login: 'user' and password: 'pass'.
Connection connection = 
  DriverManager.getConnection("jdbc:driver:db", "user", "pass");
// Prepare statement.
Statement stmt = connection.createStatement();
// Run query over data source.
ResultSet rs = 
  stmt.executeQuery("SELECT intVal, stringVal FROM TableName");
// Iterate over results.
while (rs.next()) {
    int intVal = rs.getInt("intVal");
    String stringVal = rs.getString("stringVal");
}
\end{lstlisting}
\caption{Základního použití JDBC API.}
\label{jdbc:conn}
\end{figure}


Výhodou aplikace používající JDBC API je možnost přistupovat jednotně k libovolnýmu zdoji dat na libovolné platformě, kde běží JVM. Jinak řečeno, s JDBC API není třeba pro každou odlišnou relační databázi (MySql, PostgreSQL, Oracle atd.) psát vlastní verzi aplikace, ale postačí pouze jedna aplikace využívající JDBC API. A díky JVM bude možné použít stejnou verzi aplikace i na odlišných platformách.
\subsection{Java Data Objects}
\cite{jordan:jdo,roos:jdo,tyagi:cjdo,ezzio:uujdo}
\subsection{Java Persistence API}
\cite{jpa:spec}
\subsection{Object Data Management Group}
\cite{odmg}
