V předchozí kapitole jsem se zmínil o několika nástrojích umožnující virtualizaci. Při snaze vybrat si ten nejvhodnější nás často, krom podporovaných funkcí, zajímá zejména i výkon. Ten může být u jednotlivých nástrojů v různych oblastech rozdílný. Proto vystává potřeba tyto nástroje otetsovat a následně porovnat jejich výkon u jednotlivých komponent. Mezi nejzajímavější ukazatele výkonu daného nástroje patří rychlost I/O operací (práce s disky), CPU, RAM a síťě u jeho virtualizovaných hostů (systémů). Zejména nás pak zajímá rozdíl oproti reálnému fyzickému stroji, jenž má dostupné stejné systémové prostředky, jako virtualizovaný host. Tento rozdíl ve výkonu vyjadřuje režii, která virtualizací vznikla. Čím více se výkon virtualizovaného hosta blíží výkonu reálného stroje, tím je režie nižší. Nejnižší režie rovná se nejvyšší výkon. 

Možností jak testovat výkon je více. Například velmi primitivní a nepřesné řešení je měření určité činnosti pomocí stopek. Například doba startu systému hostů v jednotlivých virtualizačních nástrojích. Když opomenu chyby vzniklé nepřesnosti měření, tak sice mohu zjistit údaj ve kterém nástroji host nejrychleji nastartoval. Ale tento údaj není přiliš vypovídající, jelikož proces nastartování celého systému je velmi komplikovaný a na celkové době se podílí výkon několika komponent. Proto například nižší výkon jedné z komponent daného nástroje, může způsobit nejdelší start jeho hosta. Ačkoliv výkon všech ostatních komponent by mohl být lepší než u nástroje, jehož host nakonec nastartoval nejrychleji.

Mnohem přesnější metodou měření je použití specializovaných nástrojů (benchmarků) zaměřujících se výhradně na testování výkonu jednotlivých komponent. Díky specializaci na určitý subsystém (CPU, RAM \dots), mají získané výsledky větší vypovídající hodnotu. A jsme z nich schopni přesněji určit, kde má daný nástroj lepší či horší výkon než jeho konkurenti. Aby výsledky byli co nejpřesnější,je třeba zajistit co nejvíce stejné prostředí a podmínky u jednotlivých hostů.

Většina těchto nástrojů dále umožňuje nastavovat různé parametry (použitý algoritmus, počet vláken, velikosti testovacích dat\dots). To umožňuje mnohem komplexnější otestování výkonu. Takže jsme o hostovi schopni zjistit nejen to, že měl nejlepší výkon v CPU testech. Ale i jak se výsledky měnili v případě použití více vláken. Nebo že je výkoný v případě sekvenčního zápisu dat na disk, ale v případě náhodného zápisu je zase pomalý atd.

\section{Nástroje pro testování výkonu}
\subsection{CPU}
\subsubsection{System Stability Tester \cite{cpu:systester}}
System Stability Tester (zkráceně Systester) je nástroj primárně určený pro testování stability procesoru po jeho přetaktování. Tento nástroj má dva režimi, jeden (test režim) je určený právě pro testování stability a druhý (bench režim) se používá pro testovní výkonu procesoru. V obou režimech se testování provádí pomocí výpočtu čísla $\pi$ na zadaný počet míst (podobný známý program je Super PI \cite{cpu:superpi}). Kde v test módu se spustí výpočet ve více vláknech (parametr \texttt{-threads}), a výsledky jednotlivých vláken se porovnají a případné rozdíly jsou vypsány na výstup. V případě bench módu se provadí pouze výpočet a jako výsledek je uveden čas potřebný k výpočtu čísla $\pi$ na daný počet míst.

\subsubsection{C-Ray}
Tato malá utilita, napsaná v jazyce C, se zaměřuje čistě na test FPU procesoru, jedná se o jednoduchý raytracing výkonostní test. Díky své jednoduchost a malé velikosti, se test na většine systému provede převážně v L1 cache procesoru. Díky tomu nejsou výsledky ovlivněny rychlostí operační paměti ani I/O operacemi \cite{cpu:cray}. Taktéž podporuje běh ve více vláknech.
\subsection{Operační paměť}
\subsubsection{RAMspeed}
RAMspeed \cite{ram:ramspeed}, jak jeho název napovídá, je nástroj zaměřující se na testování výkonu operační paměti RAM, a také pamětí cache. Je vydaný pod open source licencí, a díky tomu je v dnešní době dostupný pro velké množství architektur a operačních systémů. Existuje i ve verzi podporující více procesorů (jader). Při testování cache pamětí si můžeme určit, která sada instrukcí (jednotka procesoru) se pro testování použije. Použít můžeme buď testy vužívající ALU, FPU, MMX nebo SSE. Stejnou možnost máme i při testování výkonu paměti RAM. Zde je ještě navíc každý test složen ze čtyř podtestů (Copy, Scale, Add a Triad).
\subsubsection{SysBench}
Je multiplatformní nástroj určený k otestování výkonu systému primárně složící k otestování těch aspektů, které jsou důležité pro provoz intenzivně vytížené databáze. Ačkoliv já jej budu používat jen pro účely otestování výkonu paměťového a diskového subsystému, lze jej použít i pro testování mnoho dalších věcí. Pomocí SysBench se dá otestovat i výkon CPU, vláken a plánovače. Také zde existuje test pro otestování výkonu OLTP. Více informací v oficiální dokumentaci\footnote{\url{http://sysbench.sourceforge.net/docs/}} či na domovské stránce projektu \footnote{\url{http://sysbench.sourceforge.net}} (sekce novinky je naktuální -- aktuální verze nástroje je 0.4.12)
\subsection{I/O subsystém}
\subsubsection{Bonnie++}
Bonnie++ je program, napsaný v C++, určený k měření výkonu I/O operací disků a souborových systémů. Je založen na původním nástroji v jazyce C zvaným bonnie\footnote{\url{http://www.textuality.com/bonnie/}}. Program umožňuje dva režimi měření. V prvním režimu měří propustnost I/O subsystému, kde simuluje chování podobné při datbázovém využití. V druhém režimu vytváří, čte a maže spousty malých souborů, tím simuluje chování aplikací jako je Squid.
%\subsubsection{Seeker}
\subsubsection{Fio}
Je dalším z multiplatformních nástrojů pro testování výkonu systému a generování zátěže na něm. Veškeré informace a kód je dostupný z git repositáře\footnote{\url{http://git.kernel.dk/?p=fio.git;a=summary}}. Fio umožnujě spustit několik vláken či procesů, provádějící různé typy I/O operací, specifikované uživatelem. To se vevětšině případů provádí pomocí souboru obsahující jednotlivé instrukce (Job file), které ma nástroj provést. Fio obsahuje několik takovýchto ukázkových souborů, které umožní otestovat například podsystém sítě. Výhodou je i možnost používat tento nástroj jako server/klient. Což umožňuje spouštět testy s jiného systému, než je ten testovaný.  
\subsubsection{Netperf}
Netperf je aplikace umožňující testování výkonu související se síťovým subsystémem. Podporuje většinu Unix-ových sytému, Linux, Windows a některé další. Já jej budu využívat pro testování výkonu v TCP/IP sítích využívajících Berkeley sokety, krom nich netperf podporuje i další typy. V době psaní tohoto textu jsou to dle manuálu tyto \cite{io:netperf}:
\begin{itemize}
  \item TCP and UDP unidirectional transfer and request/response over IPv4 and IPv6 using the Sockets interface. 
  \item TCP and UDP unidirectional transfer and request/response over IPv4 using the XTI interface. 
  \item Link-level unidirectional transfer and request/response using the DLPI interface. 
  \item Unix domain sockets 
  \item SCTP unidirectional transfer and request/response over IPv4 and IPv6 using the sockets interface.
\end{itemize}
Krom propustnosti sítě je součástí výsledků měření i zátěž CPU, která byla přenosem způsobem. Jelikož je velmi obtížné toto přesně zjišťovat, je třeba brát tyto údaje s rezervou. V případě virtuálních hostů je zde navíc nepřesnost způsobená nemožností zjistit, kolik zátěže bylo vygenerováno za hranicí hosta např. v hypervizoru.
