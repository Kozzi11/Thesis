%============================================================================
% tento soubor pouzijte jako zaklad
% (c) 2008 Michal Bidlo
% E-mail: bidlom AT fit vutbr cz
%============================================================================
% kodovaní: iso-8859-2 (zmena prikazem iconv, recode nebo cstocs)
%----------------------------------------------------------------------------
% zpracování: make, make pdf, make desky, make clean
% připomínky posílejte na e-mail: bidlom AT fit.vutbr.cz
% vim: set syntax=tex encoding=latin2:
%============================================================================
\documentclass[cover]{fitthesis} % odevzdani do wisu - odkazy, na ktere se da klikat
%\documentclass[cover,print]{fitthesis} % pro tisk - na odkazy se neda klikat
%\documentclass[english,print]{fitthesis} % pro tisk - na odkazy se neda klikat
%      \documentclass[english]{fitthesis}
% * Je-li prace psana v anglickem jazyce, je zapotrebi u tridy pouzit 
%   parametr english nasledovne:
%      \documentclass[english]{fitthesis}
% * Neprejete-li si vysazet na prvni strane dokumentu desky, zruste 
%   parametr cover

% zde zvolime kodovani, ve kterem je napsan text prace
% "latin2" pro iso8859-2 nebo "cp1250" pro windows-1250, "utf8" pro "utf-8"
%\usepackage{ucs}
\usepackage[utf8]{inputenc}
\usepackage[T1]{fontenc}
\usepackage{url}
\DeclareUrlCommand\url{\def\UrlLeft{<}\def\UrlRight{>} \urlstyle{tt}}
\newcommand{\xen}{Xen\textsuperscript{\textregistered}\ }
%zde muzeme vlozit vlastni balicky


% =======================================================================
% balíček "hyperref" vytváří klikací odkazy v pdf, pokud tedy použijeme pdflatex
% problém je, že balíček hyperref musí být uveden jako poslední, takže nemůže
% být v šabloně
\ifWis
\ifx\pdfoutput\undefined % nejedeme pod pdflatexem
\else
  \usepackage{color}
  \usepackage[unicode,colorlinks,hyperindex,plainpages=false,pdftex]{hyperref}
  \definecolor{links}{rgb}{0.4,0.5,0}
  \definecolor{anchors}{rgb}{1,0,0}
  \def\AnchorColor{anchors}
  \def\LinkColor{links}
  \def\pdfBorderAttrs{/Border [0 0 0] }  % bez okrajů kolem odkazů
  \pdfcompresslevel=9
\fi
\fi

%Informace o praci/projektu
%---------------------------------------------------------------------------
\projectinfo{
  %Prace
  project=BP,            %typ prace BP/SP/DP/DR
  year=20012,             %rok
  date=\today,           %datum odevzdani
  %Nazev prace
  title.cs={Porovnání virtualizačních nástrojů\\ operačních systémů},  %nazev prace v cestine
  title.en={A comparsion of various virtualisation tool of operating systems}, %nazev prace v anglictine
  %Autor
  author={Daniel Kozák},   %jmeno prijmeni autora
  %author.title.p=Bc., %titul pred jmenem (nepovinne)
  %author.title.a=PhD, %titul za jmenem (nepovinne)
  %Ustav
  department=UITS, % doplnte prislusnou zkratku: UPSY/UIFS/UITS/UPGM
  %Skolitel
  supervisor= Jan Samek, %jmeno prijmeni skolitele
  supervisor.title.p=Ing.,   %titul pred jmenem (nepovinne)
  %supervisor.title.a={Ph.D.},    %titul za jmenem (nepovinne)
  %Klicova slova, abstrakty, prohlaseni a podekovani je mozne definovat 
  %bud pomoci nasledujicich parametru nebo pomoci vyhrazenych maker (viz dale)
  %===========================================================================
  %Klicova slova
  keywords.cs={Virtualizace, paravirtualizace, hypervizor, virtual machine monitor, virtual machine manager, VMM, virtualizace na úrovni systému, emulace, Linux-VServer, QEMU, KVM, VirtualBox, Kernel Virtual Machine, XEN, OpenVZ, UML, User Mode Linux.
}, %klicova slova v ceskem jazyce
  keywords.en={Virtualization, paravirtualization, hypervisor, virtual machine monitor, virtual machine manager, system level virtualization, emulation, Linux-VServer, QEMU, KVM, VirtualBox, Kernel Virtual Machine, XEN, OpenVZ, UML, User Mode Linux.
}, %klicova slova v anglickem jazyce
  %Abstract
  abstract.cs={V této prái se čtenář dozví co znamená a co si má představit pod pojmem virtualizace. Zjistí jaké typy virtualizace operačních systémů rozlišujeme (plná virtualizace, paravirtualizace, virtualizace na urovni systému\dots), a jaké jsou jejich výhody a nevýhody. Následně se
seznámí s open source virtualizačními stroji pro různé operační systémy. Dozví se, které typy virtualizace danný nástroj umožňuje, v jaké míře, a které platformy jsou podporovány. Následovat bude popis instalace a konfigurace jednotlivých nástrojů, způsob jejich testování a výsledky jednotlivých testů. Po přečtení by měl čtenář vedět, jaký virtualizační nástroj je nejvhodnější pro jeho potřeby.
}, % abstrakt v ceskem jazyce
  abstract.en={In this thesis, the reader learns what “virtualization” term means and what should imagine under this term. He find out which type of virtualization we distinguish and their pros and cons. Then he become familiar with open source virtualization tools for different operation systems. Next he get some information about these tools. Which platform and virtualization type they support. How to install and configure them. Then will be follow some benchmarks for these tools. So after read, one should be able to knew which virtualization tool is best choice for his needs.}, % abstrakt v anglickem jazyce
  %Prohlaseni
  declaration={Prohlašuji, že jsem tuto bakalářskou práci vypracoval samostatně pod vedením pana Ing.~Jana Samka},
  %Podekovani (nepovinne)
  %acknowledgment={Zde je možné uvést poděkování vedoucímu práce a těm, kteří poskytli odbornou pomoc.} % nepovinne
}

%Abstrakt (cesky, anglicky)
%\abstract[cs]{Do tohoto odstavce bude zapsán výtah (abstrakt) práce v českém jazyce.}
%\abstract[en]{Do tohoto odstavce bude zapsán výtah (abstrakt) práce v anglickém jazyce.}

%Klicova slova (cesky, anglicky)
%\keywords[cs]{Sem budou zapsána jednotlivá klíčová slova v českém jazyce, oddělená čárkami.}
%\keywords[en]{Sem budou zapsána jednotlivá klíčová slova v anglickém jazyce, oddělená čárkami.}

%Prohlaseni
%\declaration{Prohlašuji, že jsem tuto bakalářskou práci vypracoval samostatně pod vedením pana X...
%Další informace mi poskytli...
%Uvedl jsem všechny literární prameny a publikace, ze kterých jsem čerpal.}

%Podekovani (nepovinne)
%\acknowledgment{V této sekci je možno uvést poděkování vedoucímu práce a těm, kteří poskytli odbornou pomoc
%(externí zadavatel, konzultant, apod.).}

\begin{document}
  % Vysazeni titulnich stran
  % ----------------------------------------------
  \maketitle
  % Obsah
  % ----------------------------------------------
  \tableofcontents
  
  % Seznam obrazku a tabulek (pokud prace obsahuje velke mnozstvi obrazku, tak se to hodi)
  % \listoffigures
  % \listoftables 

  % Text prace
  % ----------------------------------------------
  \chapter{Úvod}
Objektové paradigma a rozšíření objektově orientovaných programovacích jazyků, změnilo zcela pohled na návrh a psaní programů. Narozdíl od procedurálního paradigma, do té doby nejrozšířenější paradigma, se zde místo s pojmy data, procedury a funkce, pracuje s pojmy jako je třída, objekt (a jeho stav), metoda, zpráva, dedičnost a podobně.

Tato změna pohledu programátora na návrh aplikace, přinesla i nové požadavky co se persistence aplikace týče. Do té doby zde existovala potřeba někde ukládat data, což bylo většinou zajištěno pomocí relační databáze, která se pro tyto účely perfektně hodila a poskytovala i další výhody. Například logické dělení dat do tabulek nebo efektivní vyhledávání nad daty pomocí indexů atd. Tato data byla většinou primitivní (čísla, řetězce) a dala se tedy snadno mapovat na typy podporované v dané databázi.

U objektově orientovaného programování máme místo jednoduchých dat objekty. Což jsou složité datové struktury, které mohou obsahovat další objekty ale i primitivní data. Proto zde vyvstala potřeba nástrojů umožňujících persistenci celých objektů.

Podrobnější informace o tomto problému, jak se dá řešit, jaké existují již hotové řešení a jaký je jejich výkon se Vám pokusím popsat v následujících kapitolách.


\chapter{Virtualizace}
\section{Definice}
Virtualizace je proces, při kterém se vytváří (emuluje) virtuální prostředí místo skutečného. Jako příklad uveďme virtuální pamět. Ta se z pohledu systému tváří jako celistvý blok paměti s určitou velikostí. Avšak v reálu se virtuální paměť skládá z několika různě velkých, často odlišných typů paměti. (RAM, HDD \dots).

Virtualizovat se dá skoro vše. Ať už jsou to jednotlivé komponenty počítače (síťová karta, grafika, procesor \dots) nebo také kompletně celý počítač, což je nejčastější využití a pokud nebude uvedeno jinak, tak pojmem virtualizace bude míněna právě virtualizace celého počítače.
\section{Využití a výhody}
Virtualizace přináší mnoho výhod. Díky čemuž se poslední dobou značně rozšířila, nejen ve firmách, ale i ve školách a domácnostech.

Ve firmách nachází virtualizace hlavní uplatnění na serverech. Zde roli několika samostatných fyzických strojů převzal jeden nebo pár fyzických strojů, na nemž beží několik virtuálních strojů (serverů). Hojně se také využívá možnosti spouštět na jednom stroji více (i ruzných) operačních systémů. To se zejména hodí při testování produktu, který je navržen multiplatformě. Vývojáři pak mohou aplikaci testovat zároveň na všech cílových platformách bez potřeby dalšího počítače s daným operačním systémem. Dále se virtuaůizace využívá při psokytování hostingových služeb, kde se díky virtualizaci může klient pronajmout celý virtuální počítač a ten si nechat nakonfigurovat na míru svých požadavků. Což mu dává obrovskou míru flexibility. A nejen mu, ale i společnosti, jenž mu službu nabízí, jelikož virtualizace jí umožňuje snadné přidávání a odebírání systémových prostředků.

\subsection*{Hlavní výhody virtualizace}
\begin{enumerate}
  \item plné využití dostupných prostředků fyzického stroje
  \item snížení nákladů na pořizování HW (v určitých případech i SW)
  \item zvýšení bezpečnosti díky oddělenosti jednotlivých služeb (izolace)
  \item slepšení dostupnosti a stabilty (možnost migrace)
  \item snížení rizik a zjednodušení při vývoji a testování SW
  \item větší míra flexibility (přidělování systémových prostčedků)
  \item šetření zdrojů (nejen energetických)
\end{enumerate}
\section{Virtualizace počítače}
Způsobů jak řešit virtualizaci počítače je hned několik. V následující kapitole se podíváme na základní druhy virtualizace, které se dnes pužívají.

\subsection{Virtualizace x86 architektury}
V dnešní době je x86 nejrozšířenější architekturou počítačů. Proto možnost její virtualizace je určitě lákavým cílem. Na neštěstí se ukázalo, že virtualizace x86 architektury není zrovna tou nejsnažší věcí. Důvodem je několik málo instrukcí, které musí bežet v privilegovaném režimu procesoru a předpokládají přímí přístup k fyzické paměti. Tyto instrukce by mohli změnit stav libovolného virtuálního hosta, což je nežádoucí a činní je tak problematickými.

Obecným způsobem jak tentno problém řešit je nalezení volání těchto nebezpečných instrukcí a nahradit je jinými instrukcemi, které již nejsou takto nebezpečné. V praxi se používají dva přístupi. První přístup toto nahrazování řeší přímo za běhu. A to tak, že skenuje a analyzuje běžící kód a kdykoliv narazí na volání nebezpečné instrukce tak jej nahradí emulovanou verzí. Druhý přístup, ten se používá při paravirtualizaci, nahrazuje nebezpečné instrukce už v době kompilace. To má za následek větší rychlost ale za cenu nutnosti modifikovat jádro operačního systému hosta. Další řešení problému x86 architektury je přímá HW podpora virtualizace, kterou nabízejí novější procesory jak od Intel tak i od AMD. Každé z řešení má své výhody a nevýhody. většina virtualizačních řešení v dnešní době používá některé z prvních dvou řešení v kombinaci s HW podporou virtualizace pokud je dostupná.

\subsection{Plná virtualizace}
Plná virtualizace v anglické literatuře označovaná jako \emph{full virtualization}, je jedním ze základních druhů virtualizace počítače. Yákladním principem této virtualizace je emulace kompletně celého HW počítače se všemi jeho komponenty. Na takto vytvořený plně vyrtualizovaný počítač můžeme nainstalovat libovolný operační systém, který má podporu ovladačů pro emulovaný HW. Plná virtualizace umožňuje spustit libovolný počet virtuálních strojů s odlišným OS, kde každý poběží jako vlastní proces daného virtualizačního nástroje. Díky tomu, že se emuluje kompletně celý procesor i s jeho všemi úrovněmi ochrany, není třeba řešit problematické privilegované instrukce.
\subsubsection*{Výhody}
\begin{itemize}
  \item běží zde libovolný OS bez nutnosti modifikace jádra hosta (podmínkou jsou ovladače pro emulovaný HW)
  \item isolovanost jednotlivých virtualních strojů
  \item možnost spouštět virtuální stroje s různou architekturou (ARM, x86\_64, x86)
\end{itemize}
\subsubsection*{Nevýhody}
\begin{itemize}
  \item potřeba emulace všech komponent sebou nese větší režii
\end{itemize}
\subsection{Paravirtualizace}
Se nesnaží emulovat kompletně veškerý HW, naopak využívá toho, že virtualizovaný HW se příliš neliší od toho skutečného. Všude kde je to možné, paravirtualizovaný systém přistupuje přímo ke skutečnému HW počítače. Virtualizovaný systém často nemůže využívat plný potenciál reálného fyzického HW, a taky může rozpoznat, že běží na virtuálním počítači. Paravirtualizace narozdíl od plné virtualizace musí řešit problematické privilegované operace. Toto provádí již při kompilaci. Což přináší vyšší výkon za cenu nutnosti používat modifikované jádro operačního systému hosta.
\subsubsection*{Výhody}
\begin{itemize}
  \item u některých komponent je využíván přímí přístup (nejsou zcela emulovány) $\Rightarrow$ vyšší výkon
\end{itemize}
\subsubsection*{Nevýhody}
\begin{itemize}
  \item nutnost modifikovat jádro hosta
  \item nemožnost virtualizovat systémy, ke kterým nejsou dostupné zdrojové kódy, případně neexistuje jejich varianta již s modifikovaným jádrem
\end{itemize}

\subsection{Hardwarová podpora virtualizace}
Od roku 2007 většina procesorů od firem AMD i Intel obsahuje speciální instrukční sadu (HW podporu virtualizace). U AMD procesorů se tato technologie označuje jako AMD-V dříve známá pod pojmem Pacifik. Firma Intel používá pro svoje procesory s HW virtualizací označení Intel VT. Obě tyto technologie přináší podporu další úrovně ochrany v procesoru označovanou jako Ring $-$1. Díky této nové úrovni je možné, aby jádro virtualizovaného operačního systému běželo v procesoru v ochrané úrovni RING 0. A v ochrané úrovni RING $-$1 poběží VMM (Virtual Machine Manager/Monitor), někdy označován jako hypervizor. Díky čemuž není potřeba modifikované jádro hosta. tato nová úroveň ochrany není to jediné co hardwarová podpora virtualizace přináší. Přichází s ní mnohé další vylepšení umožňující usnadnění virtualizace dalších komponent než je procesor. Například velmi užitečným rozšířeným je přidání další vrstvy pro překlad adres. Tato vrstva umožňuje mapovat lineární pamět na hostovu fyzickou paměť. Podle průzkumu od VMware HW mapování oproti čistě softwarovému řešení, přináší nárust výkonu o 42\,\%.

\subsection{Virtualizace na úrovni systému}
Zde se nejedná o plnohodnotnou virtualizaci celého počítače, ale spíše jen o virtualizaci operačního systému. Virtualizace na úrovni systému je jedním z prvních způsobů jak rozdělit výkon jednoho fyzického stroje na více nezávislých částí. A to za účelem co největšího využití potenciálu daného počítače. Funguje to tak, že systém počítače je rozdělen na několi částí zvaných kontejnery. Každý kontejner obsahuje vlastní operační systém a určité přidělené systémové prostředky. Všechny kontejnery mezi sebou sdílí stejné jádro operačního systému. Z toho vyplívají hlavní nevýhody tohoto řešení. Krom jednotného jádra jsou jinak systémy zcela oddělené a nezávislé. Tato metoda virtualizace je velmi hojně využívána zejména v hostingových společnostech. Ačkoliv s příchodem HW podpory virtualizace a čím dál lepšímu výkonu plnohodnotných virtualizačních řešení, je používání systémové virtualizace na ústupu.

\subsubsection*{Výhody}
\begin{itemize}
  \item toto řešení má téměř nulovoou režii (okolo 2-3\,\%)
  \item není potřeba HW podpora virtualizace
\end{itemize}

\subsubsection*{Nevýhody}
\begin{itemize}
  \item všechny systémy sdílí jedno jádro, pokud spadne, spadnou všechny systémy
  \item možnost instalovat jen hosty se stejným operačním systémem
  \item mírně složitější konfigurace
\end{itemize}


\chapter{Virtualizační nástroje}
\section{Xen\textsuperscript{\textregistered}}
\xen je open source VMM (\textit{\textbf{V}irtual \textbf{M}achine \textbf{M}onitor}) též zvaný jako hypervizor, podporující jak 32 bitový tak i 64 bitový x86 procesory. Je jedním z nejlepších a nejznámnějších open source řešení co se serverové virtualizace týče. Spousta profesionálních řešení s komerční podporou je vystavěna právě na základě Xen hypervizoru. Xen hypervizor běží přímo na fyzickém stroji (bare-metal) a umožňuje spouštět libovolný počet virtualizovaných strojů, které dosahují skoro stejného výkonu jako by se jednalo o nativní stroje. V následujícím textu se proto seznámíme s tímto virtualizačním řešením podrobněji.

\subsection{Historie}
Hypervizor \xen byl původně vyvinut jako součást univerzitního projektu XenoServers na Univerzitě v Cambridge. Kde hlavním cílem projektu bylo vyvinout veřejné rozhraní pro distribuovaný výpočet, jenž by bylo dostupné pro širokou veřejnost. Prvního zveřejnění se Xen hypervizor dočkal roku 2003. Jednalo se o dokument "`Xen and The Art of Virtualization"' zveřejněný na vědeckém zasedání o pricipech operačních systémů (\emph{Symposium on Operating Systems Principles}). V tomto dokukmetu je popsán hypervizor a jeho přístup řešení problému virtualizace na x86 architektuře.

Ve stejném roce vyšlo i první vydání Xen hypervizoru (verze 1.0), které bylo volně ke stažení. Od té doby se o Xen hypervizor začalo zajímat spousty firem, které použili tento hypervizor jako základ svých virtualizačních řešení. Xen byl později koupený firmou Citrix, která jej využívá pro své produkty XenServer. Samotný hypervizor však stále zůstal open source a jeho správu má na starost nadace XenSource založená jedním z původních vývojářů Ianem Prattem. Open source hypervizor se nachází na stránkách \texttt{http://www.xen.org/ }

\subsection{Architektura a způsob virtualizace}
\xen hypervizor podporuje primárně dva způsoby virtualizace stroje. První a původní je paravirtualizace, což je princip při kterém nedochází k emulaci jednotlivých komponent počítače. Namísto toho musí mít jádro hostovaného operačního systému (DomainU host) speciální ovladače pro univerzální rozhraní zařízení. Při přístupu na toto rozhraní se vlastně přistupuje nepřímo k reálnému HW. Zjednodušeně řečeno ovladač rozhraní přistupuje k ovladači (multiplexoru) v hypervizoru, který zajišťuje sdílení reálného HW mezi více hosty. Hypervizor pak následně přistupuje k reálnému HW pomocí ovladačů implementovaných v Domain0 hostu.

V předchozím odstavci jsme použil zatím nevysvětlené pojmy DomainU a Domain0 (zkráceně Dom0 a DomU). Samotný \xen hypervizor sám o sobě nemá žádné aplikační rozhraní, přes který by se dal ovládat a konfigurovat. K tomu je zapotřebí právě Dom0 host, což je v podstatě základní operační systém, který obsahuje ovladače k reálnému požitému hardware, nástroje pro práci s dalšími virtuálními hosty (DomU hosti) (vytváření, modifikování, spouštění\dots). Většinou je tímto základním Dom0 hostem některá z distribuc linuxu. Dalšími podporovanými jsou některé operační systémy Unix a BSD distribuce jako NetBSD a OpenBSD.

Dom0 host s \xen hypervizor je v podstatě minimální možná použitelná virtualizační platforma, jelikož Dom0 host obsahuje plnohodnotný operační systém, na kterém mohou bežet libovolné služby a aplikace. Tato konfigurace však nemá příliš význam, proto se jako užitečné minimum udává konfigurace s alespoň s jedním DomU hostem. Jako DomU host se označuje jakýkoliv další virtuální host bežící na \xen hypervizoru, který není Dom0 hostem. Dom0 host běží vždy jen jeden, kdežto DomU hostů může běžet teoreticky neomezeně.

Jak už jsem naznačil DomU host může běžet buď v paravirtualizováném módu nebo v HVM (\emph{\textbf{H}ardware \textbf{V}irtual \textbf{M}achine}) módu. HVM mód se od paravirtualizovaného liší tím, že již není třeba modifikovat jádro operačního systému. A virtuální host se chová zcela stejně jako by se jednalo o reálný fyzický stroj. Toto je možné jen u novějších procesorů, které mají HW podporu virtualizace. Drobnou nevýhodou je nutnost emulace, některých HW komponent virtuálního stroje. Jelikož aby se virtuální stroj choval zcela jako fyzický a operační systém nepoznal rozdíl, tak je potřeba emulovat BIOS a základní HW. Existují však i způsoby jak může takto virtualizovaný host poznat, že je virtualizován a využívat zařízení podobně jak tomu je u paravirtualizace. Toho se zejména využívá u paměťových zařízení, síťového adaptéru a PCI. Jedním z pozitivních vlastností tohoto přístupu je nižší režie. Dalším rozdílem HVM oproti paravirtualizaci je schopnost spouštět hosty i s jinačí architekturou. Například v paravirtuaůizovaném módu, pokud je Dom0 host 32bit tak i všichni DomU hosti musí bít 32bit, a nejen to dokce záleží i na dalších parametrech. Pokud by Dom0 host měl jádro s podporou PAE (\emph{\textbf{P}hysical \textbf{A}ddress \textbf{E}xtension}), tak to samé musí platit pro DomU hosty. U HVM toto neplatí, zde platí obecně pravidlo, že DomU hosti mohou být technologicky odlišní, ale nesmí být na vyšší úrovni než Dom0 host. Pokud by tedy Dom0 měl 64bit architektutu, tak DomU může být 64bit, 32bit s PAE nebo 32bit host.

\subsubsection{Jaký mód tedy zvolit?}
Obecně se dá říct, že pokud máte stroj s procesorem podporující hardwarovou virtualizaci, tak je lepší použít novější HVM mód. V případě že máte procesor bez podpory virtualizace, tak je asi nejlepším řešením paravirtualizace. Pokud dokonce máte jen 32 bitový procesor, tak je určitě dobré použít jádra s podporou PAE. To vám umožní adresovat více než 4GB operační paměti.

\subsection{Výhody a nevýhody}
\subsubsection{Výhody}
\begin{itemize}
  \item možnost virtualizace i na strojích bez podpory hardwarové virtualizace
  \item problematické instrukce řešeny již během překladu a né v době běhu $\Rightarrow$ lepší výkon
  \item hodně rozšířené, velká podpora jak komunity tak existují i komerční řešení
  \item velké množství kvalitní literatury
  \item možnost plné virtualizace v případě podpory hardwarem
\end{itemize}

\subsubsection{Nevýhody}
\begin{itemize}
  \item v případě paravirtualizovaného hosta je potřeba mít modifikované jádro
  \item hypervizor není přímo součástí jádra operačního systému
\end{itemize}

\section{Kernel-based Virtual Machine}
KVM (\emph{\textbf{K}ernel-based \textbf{V}irtual \textbf{M}achine}) je dalším výkonným open source virtualizačním řešením, které je primárně vhodné pro virtualizaci serverů. Jedná se relativně o mladé řešení, které se oproti \xen hypervisoru, ten se primárně soustředil na paravirtualizaci, dá spíše označit jako rešení podporující plnou virtualizaci. Ačkoliv si později ukážeme, že i v KVM se v určitých situacích využívá paravirtualizace. Důležitým rysem KVM je jeho závislost na HW podpoře virtualizace, bez které nelze KVM používat. Toto naštěstí v dnešní době není zas tak velký nedostatek a na většině dnešních strojů by KVM technologie měla být schopna fungovat. Původně byla podporována jen x86 architektura, ale během posledních let začala vznikat i podpora pro další architektury (powerpc, IA64, ARM \dots). Původním operačním systémem, na kterém bylo možno KVM provozovat byl operační systém s linuxovým jádrem. Dnes však existují i porty pro další jádra operačních systému jako je FreeBSD a Illumos.

\subsection{Historie}
Jak už jsem se zmínil KVM je docela mladou technologií. Do jádra linuxu se dostala teprve začátkem roku 2007 (jádro verze 2.6.20). KVM vyvinula docela neznámá firma Qumranet, a začlenění jejího kódu do jádra 2.6.20 proběhlo pro některé až velmi jednoduše. Důvodem pravděpodobně byl fakt, že si Linus Torvalds uvědomoval potřebu nativní podpory virtualizace v jádře a KVM se zdálo jako ideální řešení. V roce 2008 byla firma Qumranet odkoupena již mnohem známější firmou RedHat, Inc. Ta je nadále hlavním přispěvovatelem co se kódu KVM týče a sama na KVM staví své nové komerční virtualizační řešení. Dříve RedHat stavěl právě na již zmiňovaném \xen hypervizoru. 

\subsection{Jak to funguje}
Ačkoliv KVM a \xen hypervizor dělají v podstatě to samé, tak každý z nich to dělá mírně odlišným způsobem. KVM sám o sobě totiž není hypervizor, jelikož neobsahuje žádné ovladače ani plánovač procesů atd. Jedná se v podstatě jen o modul jádra, který vytváří rozhraní pro vytváření virtuálních hostů uvnitř jádra, jako jaderné procesy. Dalo by se tedy říct, že roli hypervizoru u KVM hraje samotné jádro. Což má hned několik výhod, jako například, že není potřeba implementovat znovu plánovač procesů, ovladače a další součásti, které by jinak hypervizor musel mít. Využívají se totiž přímo algoritmy a ovladače v jádře, to má za následek přímočarejší přístup a následné lepší využití fyzických prostředků stroje.

KVM vyžaduje hardwarovou podporu virtualizaci, což v dnešní době obsahují jak procesory firmy Intel tak i procesory od AMD. Naneštěstí oba z těchto výrobců mají vlastní implementaci a sadu instrukcí, která je odlišná od konkurenčního řešení. Tento problém je v KVM vyřešen rozdělením do více jaderných modulů. Základní \texttt{kvm} modul obsahuje jen ty části, které se napříč oběma procesory neliší a jsou obecné. Na procesoru závislé úseky kódu jsou v jednotlivých vlastních modulech. U AMD procesorů je to modul s názevem \texttt{kvm-amd} a u procesorů Intel je to modul \texttt{kvm-intel}. Po načtení kvm a \texttt{kvm-intel} nebo \texttt{kvm-amd} modulů, se vytvoří zařízení \texttt{/dev/kvm}.
Pomocí tohoto zařízení skrze volání různých \texttt{ioctl()} příkazů  probíhá veškeré ovladání (vytváření, spouštění a mazaní hsotů atd.) KVM virtualizace z uživatelského prostoru.

Důležitou součástí, bez které by nám KVM virtualizace nefungovala, je upravená verze emulátoru QEMU. Tento upravený emulátor poskytuje virtuálním hostům emulované komponenty jako je pevný disk, síťový adaptér, grafický akcelerátor atd. Na první pohled není rozdíl mezi použitím pouze QEMU, nebo KVM. Rozdíl je patrný až ve výkonu. Jelikož samotný QEMU pouze emuluje celý počítač a umožňuje na něm spouštět libovolný OS. Ale zde se jedná pouze o kompletní překlad všesch instrukcí. To znamená že veškeré činnosti provedené v takto emulovaném hostu jsou překládány a emulovány a tudíž se nespouští přímo na fyzickém procesoru. To je samozřejmě velmi pomalé hlavně co se paměťových operací týče. Teprve až kombinace QEMU a KVM přináší požadovaný výkon a dá se zde mluvit přímo o virtualizaci. Prozatím, jak jsem již psal, je třeba modifikovaná verze QEMU, která umí spolupracovat s KVM. Do budoucna je však snaha začlenit veškeré potřebné změny přímo do zdrojového kódu oficiální verze QEMU. Což by mělo usnadit a urychlit vývoj.

Z výše uvedeného textu vyplývá, že KVM virtualizace využívá plnou emulaci. To je částěčně pravda jelikož tomu tak může být. Emulace sebou však nese režii navíc, což je nežádoucí. Proto bylo vymyšleno rozhraní Virtio, které má za úkol tento problém řešit. Jedná se vlastně o podobné řešení jaké jsem již zmiňoval u \xen hypervizoru. Místo toho, aby byli veškeré komponenty plně emulovány, tak pro ty nejnáročnější na výkon (disky, GPU, síťový adaptér\dots), byli napsány speciální ovladače využívající rozhraní virtio, což je vlastně paravirtualizované rozhraní. Tudíž se operace nemusí emulovat, ale jsou pomocí virtio ovladačů přenášeny přímo na reálný hardware hostitele.

\subsection{Výhody a nevýhody}
\subsubsection{Výhody}
\begin{itemize}
  \item přímo součástí jádra operačního systému $\Rightarrow$ netřeba doinstalovávat
  \item hypervizorem je vlastně samotné jádro, to přináší menší režii a použití ověřených algoritmů a ovladačů
  \item využití již existujícíh ověřených a funkčních nástrojů (QEMU)
  \item podpora paravitualizace (Virtio)
  \item dělá jen jednu věc a pořádně
  \item rychlý vývoj
\end{itemize}
\subsubsection{Nevýhody}
\begin{itemize}
  \item poměrně nová technologie o které není pořádná literatura
  \item podpora pouze strojů s HW virtualizací (dnes už skoro irelevantní)
  %\item Virtio ovladače zatím dostupné pouze v Linuxu a Windows
\end{itemize}
\section{User Mode Linux}
\emph{\textbf{U}ser \textbf{M}ode \textbf{L}inux} (dále jen UML) je dalším zajímavým open source virtualizačním řešením. UML se od předchozích zmiňovaných technologí mírně liší. Dal by se totiž popsat spíše jako virtuální operační systém, než přímo jako virtuální stroj.

Jedná se totiž o port Linuxu na Linux, a to ve stejném smyslu jako porty Linuxu na jiné architektury procesorů (ARM, MIPS \dots). Pro lepší pochopení předchozí věty uvedu příklad. Pokud na svém stroji s x86\_64 procesorem spouštím Linux, tak jádro Linuxu volá instrukce mého procesoru (instrukční sadu x86\_64). V případě spouštění UML se jedná v podstatě o totéž. S tím rozdílem, že při spouštění UML se nevolají instrukce procesoru, ale jaderná volání. Pokud by chom tedy nahlédly do zdrojových kódů jádra Linuxu, tak v adresáři \texttt{arch}, kde jsou uloženy podporované architektury procesorů, najdem kromě adresářů \texttt{x86\_64}, \texttt{arm}~\dots, také adresář s názvem \texttt{um}.

Každý UML virtuální stroj tedy běží jako samostatný proces. To znamená, že všechny virtuální stroje jsou izolované a pád, kteréhokoliv z nich neovlivní jak hostitele, tak ani další virtuální hosty. Další výhodou je velmi snadné a rychlé nastartování a vypnutí hosta. Díky tomu jake je UML navržen, je možné hosty během jejich běhu snadno rekonfigurovat a přidělovat jim systémové prostředky. Dokonce je možné jim přidělit více RAM nebo i procesorů, než má reálně hostitel. Toho se zejména využívá při testování softwaru, u kterého potřebujem vyzkoušet funkčnost na hardwaru, který nemáme k dispozici.
\subsection{Historie}
Původním autorem UML je Jeff Dike, který začal na UML pracovat počátkem roku 1999. První veřejné oznámení se oběvilo v mailové konferenci počátkem června. Následujících několik let bylo UML vyvíjeno primárně jen Jeffem a to mimo hlavní vývojovou větev. To samozřejmně neslo několik nevýhod. Pokaždé když došlo, k některým změnám v kódu jádra. Musel Jeff tyto změny dohledávat v historii a upravovat svůj kód. Ačkoliv se z toho stala rutina, i tak si Jeff uvědomoval, že takto zabitý čas by se dal využít lépe. A to byl jeden z důvodů, proč se začal snažit protlačit svůj kód do hlavní větve. Druhým důvodem byla snaha více rozšířit UML mezi ostatní lidi.

Jeho snaha se nakonec vyplatila a 12. září 2002 byl UML začleněn do vývojové větve 2.5.34. Naneštěstí se později ukázalo, že není snadné přídávat nové funkce a opravy do hlavní větve jádra. Jelikož většina patchů, které Jeff poslal nebyla zařazena. A tak začal nacvhíli vyvíjet UML mimo hlavní větev. A to až do doby, kdy si jeden s vývojářů jádra těchto patchů všiml a nechal je zařadit do vývojové větve Andrew Mortona (Andrew Morton byl druhý ve vedení hned po Linusu Torvaldsovi). Zde UML zůstalo po určitou dobu a získalo si spoustu příznivců. Takže se nakonec opět dostalo do hlavní větve jádra (2.6.9) a od té doby je jeho součástí.

Prvních 5 let Jeff pracoval na UML jen ve svém volném čase a živil se jako IT konzultant. To se změnilo v roce 2004, kdy dostal pracovní nabídku od firem Intel a RedHat, Inc. Obě firmy mu nabídli možnost nadále pokračovat ve vývoji UML a Jeff nakonec začal pracovat pro Intel, kde se od roku 2004 nadále věnuje vývoji UML.

\subsection{Výhody a nevýhody}
\subsubsection{Výhody}
\begin{itemize}
  \item každý virtuální stroj běží jako samostatný proces $\rightarrow$ izolace
  \item možnost přidělovat více prostředků než je reálně dostupné
  \item přímo součástí jádra
  \item rychlé zapínání a vypínaní virtuálních hostů
\end{itemize}

\subsubsection{Nevýhody}
\begin{itemize}
  \item nejedná se o plnohodnotnou virtualizaci, což má určitá omezení
  \item nemožnost použití nemodifikovaných hostů
  \item podpora pouze Linux hostů
  \item né moc skvělý výkon, zatím chybí podpora HW virtualizace
\end{itemize}

\section{VirtualBox}

\section{LXC}
\section{Linux-VServer}


\chapter{Testovací prostředí}
V následující kapitole si popíšeme něco málo o sestavě, která byla pro účely testování a srovnáváni výkonu virtualizačních nástrojů použita. Krom hardwarové výbavy testovacího stroje, což je základní minimum, které by mělo být uvedeno v každé práci zabývající se testováním výkonu určitých softwarových produktů, si také uvedeme operační systém a jeho přesnou verzi. Pro zvýšení vypovídající hodnoty následujících testů, zde uvedu i přesné verze jednotlivývh nástrojů použitých při testování.

\section{Testovací sestava}
\subsection{Základní deska}
Základem celého počítače je deska 880GMH/U3S3 od firmy ASRock. Důvodem pro její zvolení byla údajná podpora nejnovější rodiny AMD procesorů Bulldozer a také její podpora taktování. Během osazení této desky jednim z nových procesorů, se ukázalo, že podporu bulldozerů sice deska má. Bohužel až s nejnovější verzí BIOSu. Tu samozřejmě většina prodávaných kusů neměla. Bylo tedy nutno použít starší AMD procesor se socketem AM2+ a vyšší. A následně provést update BIOSu. Poté už vše fungovalo v pořádku.
\subsubsection{Přehled kompletních parametrů}
\begin{itemize}
  \item Podpora pro AM3+ procesory, až 8 jader
  \item 100\,\% pevných kondezátorů (delší životnost) 
  \item Podpora pro Dual Channel DDR3 2000(OC)
  \item Podpora ATI™ Hybrid CrossFireX™
  \item Integrovaná GPU AMD Radeon HD 4250 graphics, DX10.1 class iGPU, Shader Model 4.1
  \item Video výstupy : D-Sub, DVI-D a HDMI
  \item 2 x USB 3.0, 2 x SATA3, C.C.R. (Combo Cooler Retention Module)
  \item Podpora dalších technologií: AXTU, Graphical UEFI, Instant Boot, Instant Flash, APP Charger, SmartView
\end{itemize}

\subsection{Procesor}
Jak už jsme se výšše zmínil, použit byl procesor od firmy AMD se socketem AM3+ . Jednalo se o jeden z prvních modelů nové rodiny Bulldozer. Přesné označení procesoru je AMD FX-4100. Jedná se o procesor se čtyřmi jádry, kde každý z nich je v základním nastavení taktován na frekvenci 3,6 GHz (respektive 3,7 -- 3,8 v Turbo módu). Díky odemknutému násobiči je tento procesor přímo předurčen pro přetaktování. V době testů byl však ponechán na jeho základních frekvencí s vypnutým turbo boostem.

\subsubsection{Podrobné parametry}
\begin{itemize}
  \item Socket: AM3+
  \item L2 cache: 4 MB
  \item Frekvence: 3600 MHz
  \item Typ: AMD FX
  \item Model: FX-4100
  \item Počet jader: 4
  \item Operační módy: 32-bit a 64-bit
  \item L3 cache: 8 MB
  \item Výrobní proces: 32\,nm
  \item TDP: 95\,W
\end{itemize}

\subsection{Disky}
Počítačová sestava je vybavena dvěma pevnými disky jedním rychlým systémovým SSD diskem Verbatim o kapacitě 128\,GB, a druhým datovým SATA-II diskem Samsung o kapacitě 1\,GB.
\subsubsection{Parametry systémového disku}
\begin{itemize}
  \item Výrobce: Verbatim
  \item Model: SSD Black Edition
  \item Kapacita: 128\,GB
  \item Rozhraní: Serial ATA II
  \item Průměrný vyhledávací čas: 0.1\,ms
  \item Rychlost čtení: 270\,MB/s
  \item Rychlost zápisu: 225\,MB/s
  \item Rychlost přenosu dat: 3\,Gbit/s
  \item Speciální funkce: podpora TRIM a NCQ
\end{itemize}

\subsubsection{Parametry datového disku}
\begin{itemize}
  \item Výrobce: Samsung
  \item Model: SpinPoint F1
  \item Kapacita: 1000\,GB 
  \item Průměrný vyhledávací čas: 8.9\,ms  
  \item Rychlost otáčení ploten: 7200\,rpm
  \item Vyrovnávací paměť: 32\,MB
  \item Rozhraní: Serial ATA II
  \item Rychlost přenosu dat: 3\,Gbit/s
  \item Speciální funkce: NCQ
\end{itemize}

\subsection{Operační paměť}
Celkově počítač obsahuje čtyři paměťové moduly, kde každý má veliost 4\,GB a jsou spárovány po dvou, tak aby bylo využito podpory dual-channel. Dohromady tedy sestava disponuje 16\,GB RAM. Což je pro testovací účely víc než dostačující.
\section{Operační systém}
\subsection{Zvolená platforma}
První věcí, kterou bylo třeba při výběru OS rozhodnou byla platforma potažmo jádro operačního systému. Jelikož několik zvolených virtualizačních nástrojů podporuje pouze Linux. Nebyl důvod se moc rozmýšlet. Chvíli jsem uvažoval o některé variantě BSD systému, jelikož by zde šla teoreticky rozchodit většina zvolených nástrojů, a ty které by zde nešli, mají svoji alternativu. Ale nakonec jsem tuto myšlenku potlačil, jelikož Linux je rozšířenější, a dá se předpokládat, že pokud někdo bude některý ze zmiňovaných virtualizačních nástrojů nasazovat, tak to bude právě na linuxu.

\subsection{Volba distribuce}
Dalším krokem po vybrání linuxu jako hlavní platformy, na níž budu provádět veškeré testování, byla volba distribuce. Zde to již bylo o něco složitější, jelikož existuje skoro bezpočet distribucí, kde každá má své klady a zápory.

 

\section{Verze virtualizačních nástrojů}


\chapter{Výkonnostní testy}
\section{Metodika měření, aneb co a jak se bude meřit}
\section{Výsledky měření}


\chapter{Závěr}
Mým cílem v této práci bylo seznámit čtenáře s problematikou persistence objektů a jejich stavů v OOP jazycích, zejména v programovacím jazyce Java. Ale i přes zaměření na tento programovací jazyk, jsou informace uvedené v této práci platné z velké části i pro ostatní OOP jazyky. Toto platí zejména pro první dvě teoretické kapitoly.

Ve zbylých kapitolách jsem se více zaměřil na popis jazyka Javy a jednotlivých nástrojů a specifikací, jenž jsou v tomto jazyce využívány pro persistenci dat.

Během psaní této práce, jsem si rozšířil svoje znalosti o hodně zajímavých informací. A během implementace podpůrného programu, jenž jsem využil pro testování výkonu jednotlivých nástrojů, jsem narazil na několik zajímavých problémů. A to zejména na fakt, že specifikace je jedna věc, ale implementace je věc druhá.

Ačkoliv všechny mnou testované ORM nástroje by měli splňovat JPA specifikaci, jen málo kdy se podařilo napsat kód využívající JPA API tak, aby běžel ve všech mnou testovaných implementacích.

Asi nejméně problematickými nástroji byly v tomto směru Hibernate a EclipseLink. Největší potíže jsem měl s DataNucleus, u kterého se mi jeden test nepodařilo implementovat vůbec.

Docela zábavná byla i situace, kdy jedna implementace (BatooJPA) měla prohozený význam funkcí \texttt{EMPTY} a \texttt{NOT EMPTY}. Což na můj popud bylo vývojáři rychle opraveno, ale bohužel jen částečně. Sice opravili toto chování v kódu, ale při snaze zkompilovat opravenou verzi, to zhavarovalo na jednotkových testech, kde se stále počítalo se špatným chováním těchto funkcí.

Naštěstí se jedná o otevřený software, jenž má svůj kód dostupný na githubu, takže nebyl problém si vytvořit kopii repositáře a opravit rozbité testy. Na můj požadavek začlenit tuto opravu bylo zareagováno opravdu rychle. To hodnotím velmi pozitivně. Dokonce jsem během testování narazil i na pár dalších chyb, které jsem touto cestou odstranil. Takže bylo nakonec možné tuto implementaci zařadit do mého testování, tedy až na výjimku PostgreSQL databáze, s touto databází se mi testy nepodařilo provést a na nalezení a opravení chyby mi již nezbyl čas.

Další zajímavý problém jsem měl s OpenJPA implementací. Zde se sice nejednalo o problém nesouladu specifikace a implementace, ale o výkonnostní potíže při určitém typu operací. To některé mé testy degradovalo natolik, že jejich provedení trvalo místo obvyklých pár vteřin i několik desítek minut. Naštěstí pár dní před odevzdáním této práce vyšla aktualizovaná verze této implementace. Kde se již naštěstí tento problém nevyskytoval, takže výsledky uvedené v páté kapitole nejsou tímto nijak poznamenány.

Porovnání jednotlivých nástrojů bylo zaměřeno na jejich výkon, tedy na rychlost jednotlivých operací. Což je jeden z nejzákladnějších a nejzajímavějších faktorů při volbě daného nástroje. Na druhou stranu zde existuje prostor pro rozšíření práce o měření dalších faktorů. A to zejména měření nároků na systémové prostředky jako je procesor, operační paměť nebo velikost zabraného místa na disku.

Jelikož veškeré testy jsem prováděl s výchozím nastavením jednotlivých nástrojů. Je zde i možnost zaměřit se na optimalizaci nastavení jednotlivých nástrojů a databází. Ale rozsah tohoto téma by vydal na samostatnou práci. 

%Ve čtvrté kapitole jsem se snažil čtenáře seznámit s programovacím jazykem Java a popsat mu %některé techniky a vlastnosti jazyka, jenž jsou používané různými nástroji pro podporu persistence objektů. 

%Dále jsem v téže kapitole popsal existující nástroje a specifikace, které ve světě jazyka Java existují a řeší problémy persistence dat či objketů, objektově relační mapování, transakční zpracování a mnoho dalších souvisejících ůloh.

%V posledních dvou kapitolách jsem se zaměřil na výkonnostní testování některých existujících nástrojů pro persistenci objektů. Tyto nástroje jsem rozdělil do dvou skupin. Kde první skupinou byly objektové databáze a druhou skupinou byly nástroje mapující objekty na relační databáze.

%Tyto jednotlivé nástroje a databáze jsem otestoval pomocí mnou napsané aplikace, která testovala výkonost jednotlivých řešení pomocí sady testů. Výsledky těchto testů jsem následně 

 % viz. obsah.tex

  % Pouzita literatura
  % ----------------------------------------------
\ifczech
  \bibliographystyle{czechiso}
\else 
  \bibliographystyle{plain}
%  \bibliographystyle{alpha}
\fi
  \begin{flushleft}
  \bibliography{literatura} % viz. literatura.bib
  \end{flushleft}
  \appendix
  
  \chapter{\xen příprava balíčku}
\section{PKGBUILD}

\begin{lstlisting}
pkgname=xen-rc
pkgver=23276
pkgrel=1
pkgdesc="Xen 4.1.3 rc (hypervisor tools and doc) HG"
arch=(i686 x86_64)
url="http://xen.org/"
license="GPL"
depends=('bzip2' 'iproute' 'bridge-utils' 'python2' 'sdl'
         'zlib' 'e2fsprogs' 'bin86' 'pkgconfig' 'gnutls'
         'lzo2' 'glibc')
makedepends=('dev86' 'mercurial' 'git' 'ghostscript')
conflicts=('xen' 'xen3' 'xen4' 'xen-hv-tools' 'libxen4')
provides=('xen')
source=(archinit.patch texi2html.patch 09_xen)

md5sums=('d3ab9bbae472e613a04dc8e62377ed93'
         'c94602f1feaa5d968db1e9f640dfd2a5'
         '3a3240e1edde3a8e295f928be52dbde4')
			   
_hgroot="http://xenbits.xensource.com/"
_hgrepo="xen-4.1-testing.hg"

build() {

  cd "$srcdir"
  msg "Connecting to Mercurial server...."

  if [ -d $_hgrepo ] ; then
    cd $_hgrepo
    hg pull -u || return 1
    msg "The local files are updated."
  else
    hg clone $_hgroot $_hgrepo || return 1
  fi

  msg "Mercurial checkout done or server timeout"
  msg "Starting make..."

  rm -rf "$srcdir/$_hgrepo-build"
  cp -r "$srcdir/$_hgrepo" "$srcdir/$_hgrepo-build"
  cd "$srcdir/$_hgrepo-build"

  patch -p1 -F99 -i ../archinit.patch
  patch -p1 -i ../texi2html.patch
  unset CFLAGS LDFLAGS

  make PYTHON=python2 DESTDIR=$pkgdir  xen
  make PYTHON=python2 DESTDIR=$pkgdir  tools  
    
}

package() {

  cd "$srcdir/$_hgrepo-build"
  unset CFLAGS LDFLAGS
  make PYTHON=python2 DESTDIR=$pkgdir  install-xen
  make PYTHON=python2 DESTDIR=$pkgdir  install-tools  
  
  sed -i 's#XENDOM_CONFIG=/etc/sysconfig/xendomains#XENDOM_CONFIG=/etc/conf.d/xendomains#' $pkgdir/etc/init.d/xendomains
  sed -i "s#touch /var/lock/subsys/xend#mkdir -p /var/lock/subsys\n     touch /var/lock/subsys/xend#" $pkgdir/etc/init.d/xend

  [ -d $pkgdir/usr/lib64 ] && ( cd $pkgdir/usr && cp -R lib64/* lib/ && rm -R lib64 )
  ( cd $pkgdir/etc && mv init.d rc.d ) || return 1
  rm -f $pkgdir/usr/share/man/man1/qemu-img.1* \
       $pkgdir/usr/share/man/man1/qemu.1*
  # First experiment to generate grub2.cfg entry
  mkdir -p $pkgdir/etc/grub.d
  chmod +x $srcdir/09_xen
  cp $srcdir/09_xen  $pkgdir/etc/grub.d

  ############ kill unwanted stuff ############
  # stubdom: newlib
  rm -rf $pkgdir/usr/*-xen-elf

  # hypervisor symlinks
  rm -rf $pkgdir/boot/xen-4.1.gz
  rm -rf $pkgdir/boot/xen-4.gz
  rm -rf $pkgdir/boot/xen.gz

  # silly doc dir fun
  rm -fr $pkgdir/usr/share/doc/xen
  rm -rf $pkgdir/usr/share/doc/qemu

  # Pointless helper
  rm -f $pkgdir/usr/sbin/xen-python-path

  # qemu stuff (unused or available from upstream)
  rm -rf $pkgdir/usr/share/xen/man
  rm -rf $pkgdir/usr/bin/qemu-*-xen
  for file in bios.bin openbios-sparc32 openbios-sparc64 ppc_rom.bin \
         pxe-e1000.bin pxe-ne2k_pci.bin pxe-pcnet.bin pxe-rtl8139.bin \
         vgabios.bin vgabios-cirrus.bin video.x openbios-ppc bamboo.dtb
  do
        rm -f $pkgdir/usr/share/xen/qemu/$file
  done

  # adhere to Static Library Packaging Guidelines
  rm -rf $pkgdir/usr/lib/*.a 	
}
\end{lstlisting}

\section{archinit.patch}
\begin{lstlisting}
diff -Naur orig.xen-4.1.1//tools/hotplug/Linux/init.d/xencommons xen-4.1.1//tools/hotplug/Linux/init.d/xencommons
--- orig.xen-4.1.1//tools/hotplug/Linux/init.d/xencommons	2011-07-03 03:08:44.953747064 -0700
+++ xen-4.1.1//tools/hotplug/Linux/init.d/xencommons	2011-07-05 13:47:54.627029164 -0700
@@ -18,6 +18,9 @@
 # Description:       Starts and stops the daemons neeeded for xl/xend
 ### END INIT INFO
 
+. /etc/rc.conf
+. /etc/rc.d/functions
+
 if [ -d /etc/sysconfig ]; then
 	xencommons_config=/etc/sysconfig
 else
@@ -26,7 +29,7 @@
 
 test -f $xencommons_config/xencommons && . $xencommons_config/xencommons
 
-XENCONSOLED_PIDFILE=/var/run/xenconsoled.pid
+XENCONSOLED_PIDFILE=/run/daemons/xenconsoled.pid
 shopt -s extglob
 
 if test "x$1" = xstart && \
@@ -60,33 +64,39 @@
 		    time=$(($time+1))
                     sleep 1
                 done
-		echo
-
 		# Exit if we timed out
 		if ! [ $time -lt $timeout ] ; then
-		    echo Could not start xenstored
+		    #echo Could not start xenstored
+                    stat_fail
 		    exit 1
 		fi
+                stat_done
 
-		echo Setting domain 0 name...
+		stat_busy "Setting domain 0 name..."
 		xenstore-write "/local/domain/0/name" "Domain-0"
+		stat_done
 	fi
 
-	echo Starting xenconsoled...
+	#echo Starting xenconsoled...
+	stat_busy "Starting xenconsoled"
 	test -z "$XENCONSOLED_TRACE" || XENCONSOLED_ARGS=" --log=$XENCONSOLED_TRACE"
 	xenconsoled --pid-file=$XENCONSOLED_PIDFILE $XENCONSOLED_ARGS
 	test -z "$XENBACKENDD_DEBUG" || XENBACKENDD_ARGS="-d"
 	test "`uname`" != "NetBSD" || xenbackendd $XENBACKENDD_ARGS
+	stat_done
+	add_daemon xencommons
 }
 do_stop () {
-        echo Stopping xenconsoled
+         stat_busy "Stopping xenconsoled"
 	if read 2>/dev/null <$XENCONSOLED_PIDFILE pid; then
 		kill $pid
 		while kill -9 $pid >/dev/null 2>&1; do sleep 0.1; done
 		rm -f $XENCONSOLED_PIDFILE
 	fi
+	stat_done
 
-	echo WARNING: Not stopping xenstored, as it cannot be restarted.
+	printhl "WARNING: Not stopping xenstored, as it cannot be restarted."
+        rm_daemon xencommons
 }
 
 case "$1" in
diff -Naur orig.xen-4.1.1//tools/hotplug/Linux/init.d/xend xen-4.1.1//tools/hotplug/Linux/init.d/xend
--- orig.xen-4.1.1//tools/hotplug/Linux/init.d/xend	2011-07-03 03:08:44.953747064 -0700
+++ xen-4.1.1//tools/hotplug/Linux/init.d/xend	2011-07-05 01:47:40.981951191 -0700
@@ -18,6 +18,10 @@
 # Description:       Starts and stops the Xen control daemon.
 ### END INIT INFO
 
+. /etc/rc.conf
+. /etc/rc.d/functions
+
+
 shopt -s extglob
 
 # Wait for Xend to be up
@@ -37,23 +41,30 @@
 case "$1" in
   start)
 	if [ -z "`ps -C xenconsoled -o pid=`" ]; then
-		echo "xencommons should be started first."
+	 printhl "xencommons should be started first."
 		exit 1
 	fi
 	# mkdir shouldn't be needed as most distros have this already created. Default to using subsys.
 	# See docs/misc/distro_mapping.txt
-	mkdir -p /var/lock
-	if [ -d /var/lock/subsys ] ; then
-		touch /var/lock/subsys/xend
+	if [ -d /run/lock/subsys ] ; then
+		touch /run/lock/subsys/xend
 	else
-		touch /var/lock/xend
+		touch /run/lock/xend
 	fi
+	stat_busy "Starting xend"
 	xend start
 	await_daemons_up
+	stat_done
+	add_daemon xend
 	;;
+
+
   stop)
+   stat_busy "Stopping xend"
 	xend stop
-	rm -f /var/lock/subsys/xend /var/lock/xend
+	rm -f /run/lock/xend /var/lock/xend
+	stat_done
+	rm_daemon xend
 	;;
   status)
 	xend status
@@ -62,8 +73,10 @@
         xend reload
         ;;
   restart|force-reload)
+   stat_busy "Restarting xend"
 	xend restart
 	await_daemons_up
+	stat_done
 	;;
   *)
 	# do not advertise unreasonable commands that there is no reason
diff -Naur orig.xen-4.1.1//tools/hotplug/Linux/init.d/xendomains xen-4.1.1//tools/hotplug/Linux/init.d/xendomains
--- orig.xen-4.1.1//tools/hotplug/Linux/init.d/xendomains	2011-07-03 03:08:44.953747064 -0700
+++ xen-4.1.1//tools/hotplug/Linux/init.d/xendomains	2011-07-05 13:46:36.208222760 -0700
@@ -26,6 +26,9 @@
 # Description:       Start / stop domains automatically when domain 0 
 #                    boots / shuts down.
 ### END INIT INFO
+. /etc/rc.conf
+. /etc/rc.d/functions
+
 
 CMD=xm
 $CMD list &> /dev/null
@@ -46,93 +49,52 @@
 	exit 0
 fi
 
-# See docs/misc/distro_mapping.txt
-if [ -d /var/lock/subsys ]; then
-	LOCKFILE=/var/lock/subsys/xendomains
-else
-	LOCKFILE=/var/lock/xendomains
-fi
-
-if [ -d /etc/sysconfig ]; then
-	XENDOM_CONFIG=/etc/sysconfig/xendomains
-else
-	XENDOM_CONFIG=/etc/default/xendomains
-fi
+LOCKFILE=/run/lock/xendomains
+XENDOM_CONFIG=/etc/default/xendomains
 
-test -r $XENDOM_CONFIG || { echo "$XENDOM_CONFIG not existing";
+test -r $XENDOM_CONFIG || { 
+	printhl "$XENDOM_CONFIG not existing";
 	if [ "$1" = "stop" ]; then exit 0;
 	else exit 6; fi; }
 
 . $XENDOM_CONFIG
 
-# Use the SUSE rc_ init script functions;
-# emulate them on LSB, RH and other systems
-if test -e /etc/rc.status; then
-    # SUSE rc script library
-    . /etc/rc.status
-else    
-    _cmd=$1
-    declare -a _SMSG
-    if test "${_cmd}" = "status"; then
+_cmd=$1
+declare -a _SMSG
+if test "${_cmd}" = "status"; then
 	_SMSG=(running dead dead unused unknown)
 	_RC_UNUSED=3
-    else
+else
 	_SMSG=(done failed failed missed failed skipped unused failed failed)
 	_RC_UNUSED=6
-    fi
-    if test -e /etc/init.d/functions; then
-	# REDHAT
-	. /etc/init.d/functions
-	echo_rc()
-	{
-	    #echo -n "  [${_SMSG[${_RC_RV}]}] "
-	    if test ${_RC_RV} = 0; then
-		success "  [${_SMSG[${_RC_RV}]}] "
-	    else
-		failure "  [${_SMSG[${_RC_RV}]}] "
-	    fi
-	}
-    elif test -e /lib/lsb/init-functions; then
-	# LSB    
-    	. /lib/lsb/init-functions
-        if alias log_success_msg >/dev/null 2>/dev/null; then
-	  echo_rc()
-	  {
-	       echo "  [${_SMSG[${_RC_RV}]}] "
-	  }
-        else
-	  echo_rc()
-	  {
-	    if test ${_RC_RV} = 0; then
-		log_success_msg "  [${_SMSG[${_RC_RV}]}] "
-	    else
-		log_failure_msg "  [${_SMSG[${_RC_RV}]}] "
-	    fi
-	  }
-        fi
-    else    
-	# emulate it
-	echo_rc()
-	{
-	    echo "  [${_SMSG[${_RC_RV}]}] "
-	}
-    fi
-    rc_reset() { _RC_RV=0; }
-    rc_failed()
-    {
+fi
+
+
+
+echo_rc() {
+	echo
+	printhl "Return Status: ${_SMSG[${_RC_RV}]}"
+}
+
+
+rc_reset() { _RC_RV=0; }
+
+
+rc_failed() {
 	if test -z "$1"; then 
-	    _RC_RV=1;
+		_RC_RV=1;
 	elif test "$1" != "0"; then 
-	    _RC_RV=$1; 
-    	fi
+		_RC_RV=$1; 
+	fi
 	return ${_RC_RV}
-    }
-    rc_check()
-    {
+}
+
+rc_check() {
 	return rc_failed $?
-    }	
-    rc_status()
-    {
+}	
+
+
+rc_status() {
 	rc_failed $?
 	if test "$1" = "-r"; then _RC_RV=0; shift; fi
 	if test "$1" = "-s"; then rc_failed 5; echo_rc; rc_failed 3; shift; fi
@@ -140,26 +102,24 @@
 	if test "$1" = "-v"; then echo_rc; shift; fi
 	if test "$1" = "-r"; then _RC_RV=0; shift; fi
 	return ${_RC_RV}
-    }
-    rc_exit() { exit ${_RC_RV}; }
-    rc_active() 
-    {
+}
+
+
+rc_exit() { exit ${_RC_RV}; }
+
+
+rc_active() {
 	if test -z "$RUNLEVEL"; then read RUNLEVEL REST < <(/sbin/runlevel); fi
 	if test -e /etc/init.d/S[0-9][0-9]${1}; then return 0; fi
 	return 1
-    }
-fi
+}
 
-if ! which usleep >&/dev/null
-then
-  usleep()
-  {
-    if [ -n "$1" ]
-    then
-      sleep $(( $1 / 1000000 ))
-    fi
-  }
-fi
+usleep() {
+	if [ -n "$1" ]
+	then
+	  sleep $(( $1 / 1000000 ))
+	fi
+}
 
 # Reset status of this service
 rc_reset
@@ -235,10 +195,12 @@
 start() 
 {
     if [ -f $LOCKFILE ]; then 
-	echo -e "xendomains already running (lockfile exists)"
+	stat_busy "xendomains already running (lockfile exists)"
+	stat_fail
 	return; 
     fi
 
+    printhl "Starting Xen Domains"
     saved_domains=" "
     if [ "$XENDOMAINS_RESTORE" = "true" ] &&
        contains_something "$XENDOMAINS_SAVE"
@@ -299,6 +261,7 @@
 	    fi
 	done
     fi
+    add_daemon xendomains
 }
 
 all_zombies()
@@ -352,7 +315,7 @@
     if test "$XENDOMAINS_AUTO_ONLY" = "true"; then
 	rdnames
     fi
-    echo -n "Shutting down Xen domains:"
+    printhl "Shutting down Xen domains"
     name=;id=
     while read LN; do
 	parseln "$LN" || continue
@@ -465,6 +428,7 @@
     rm -f $LOCKFILE
     
     exec 2>&3
+    rm_daemon xendomains
 }
 
 check_domain_up()
diff -Naur orig.xen-4.1.1//tools/hotplug/Linux/init.d/xen-watchdog xen-4.1.1//tools/hotplug/Linux/init.d/xen-watchdog
--- orig.xen-4.1.1//tools/hotplug/Linux/init.d/xen-watchdog	2011-07-03 03:08:44.957080397 -0700
+++ xen-4.1.1//tools/hotplug/Linux/init.d/xen-watchdog	2011-07-05 13:20:22.515289867 -0700
@@ -17,49 +17,32 @@
 ### END INIT INFO
 #
 
+. /etc/rc.conf
+. /etc/rc.d/functions
+
 DAEMON=/usr/sbin/xenwatchdogd
 base=$(basename $DAEMON)
+initname="xen-watchdog"
 
-# Source function library.
-if [ -e  /etc/init.d/functions ] ; then
-    . /etc/init.d/functions
-elif [ -e /lib/lsb/init-functions ] ; then
-    . /lib/lsb/init-functions
-    success () {
-        log_success_msg $*
-    }
-    failure () {
-        log_failure_msg $*
-    }
-else
-    success () {
-        echo $*
-    }
-    failure () {
-        echo $*
-    }
-fi
 
 start() {
 	local r
-	echo -n $"Starting domain watchdog daemon: "
+	stat_busy "Starting domain watchdog daemon"
 
 	$DAEMON 30 15
 	r=$?
-	[ "$r" -eq 0 ] && success $"$base startup" || failure $"$base startup"
-	echo
+	[ "$r" -eq 0 ] && stat_done ; add_daemon $initname || stat_fail
 
 	return $r
 }
 
 stop() {
 	local r
-	echo -n $"Stopping domain watchdog daemon: "
+	stat_busy "Stopping domain watchdog daemon"
 
 	killall -USR1 $base 2>/dev/null
 	r=$?
-	[ "$r" -eq 0 ] && success $"$base stop" || failure $"$base stop"
-	echo
+	[ "$r" -eq 0 ] && stat_done ; rm_daemon $initname || stat_fail
 
 	return $r
 }

\end{lstlisting}

\section{texi2html.patch}
\begin{lstlisting}
diff -Naur xen-4.1-testing.hg.orig/tools/Makefile xen-4.1-testing.hg/tools/Makefile
--- xen-4.1-testing.hg.orig/tools/Makefile	2012-03-18 09:32:20.974961585 +0100
+++ xen-4.1-testing.hg/tools/Makefile	2012-03-18 09:07:37.000000000 +0100
@@ -107,6 +107,7 @@
 	set -e; \
 		$(buildmakevars2shellvars); \
 		cd ioemu-dir; \
+		sed -i 's/number[ ]/number-sections /' Makefile; \
 		$(QEMU_ROOT)/xen-setup $(IOEMU_CONFIGURE_CROSS)
 
 .PHONY: ioemu-dir-force-update
\end{lstlisting}

\section{09\_xen}
\begin{lstlisting}
#! /bin/sh -e

if [ -f /usr/lib/grub/grub-mkconfig_lib ]; then
  . /usr/lib/grub/grub-mkconfig_lib
else
  # no grub file, so we notify and exit gracefully
  echo "Cannot find grub config file, exiting." >&2
  exit 0
fi

XEN_HYPERVISOR_CMDLINE="xsave=1"
XEN_LINUX_CMDLINE="console=tty0"
[ -r /etc/xen/grub.conf ] && . /etc/xen/grub.conf

CLASS="--class gnu-linux --class gnu --class os"

if [ "x${GRUB_DISTRIBUTOR}" = "x" ] ; then
  OS=GNU/Linux
else
  OS="${GRUB_DISTRIBUTOR} GNU/Linux"
  CLASS="--class $(echo ${GRUB_DISTRIBUTOR} | tr '[A-Z]' '[a-z]' | cut -d' ' -f1) ${CLASS}"
fi

# loop-AES arranges things so that /dev/loop/X can be our root device, but
# the initrds that Linux uses don't like that.
case ${GRUB_DEVICE} in
  /dev/loop/*|/dev/loop[0-9])
    GRUB_DEVICE=`losetup ${GRUB_DEVICE} | sed -e "s/^[^(]*(\([^)]\+\)).*/\1/"`
  ;;
esac

if [ "x${GRUB_DEVICE_UUID}" = "x" ] || [ "x${GRUB_DISABLE_LINUX_UUID}" = "xtrue" ] \
    || ! test -e "/dev/disk/by-uuid/${GRUB_DEVICE_UUID}" \
    || [ "`grub-probe -t abstraction --device ${GRUB_DEVICE} | sed -e 's,.*\(lvm\).*,\1,'`" = "lvm"  ] ; then
  LINUX_ROOT_DEVICE=${GRUB_DEVICE}
else
  LINUX_ROOT_DEVICE=UUID=${GRUB_DEVICE_UUID}
fi

xen_entry ()
{
  os="$1"
  xen_version="$2"
  version="$3"
  xen_args="$4"
  args="$5"
  printf "menuentry 'Xen %s / %s, with Linux %s' --class xen ${CLASS} {\n" "${xen_version}" "${os}" "${version}"
  save_default_entry | sed -e "s/^/\t/"

  if [ -z "${prepare_boot_cache}" ]; then
    prepare_boot_cache="$(prepare_grub_to_access_device ${GRUB_DEVICE_BOOT} | sed -e "s/^/\t/")"
  fi
  printf '%s\n' "${prepare_boot_cache}"
  cat << EOF
       echo    '$(printf "Loading Xen %s ..." ${xen_version})'
       multiboot       ${rel_dirname}/${xen_basename} ${rel_dirname}/${xen_basename} ${xen_args}
       echo    $(printf "$(gettext "Loading Linux %s ...")" ${version})
       module  ${rel_dirname}/${basename} ${rel_dirname}/${basename} root=${linux_root_device_thisversion} ro ${args}
EOF
  if test -n "${initrd}" ; then
    cat << EOF
       echo    "Loading initial ramdisk ..."
       module  ${rel_dirname}/${initrd}
EOF
  fi
  cat << EOF
}
EOF
}

xen_list=`for i in /boot/xen-*.gz /xen-*.gz ; do
       if grub_file_is_not_garbage "$i" ; then echo -n "$i "; fi
done`
prepare_boot_cache=

while [ "x$xen_list" != "x" ] ; do
  xen=`version_find_latest $xen_list`
  echo "Found Xen hypervisor image: $xen" >&2
  xen_basename=`basename $xen`
  xen_dirname=`dirname $xen`
  rel_xen_dirname=`make_system_path_relative_to_its_root $xen_dirname`
  xen_version=`echo $xen_basename | sed -e "s,^[^0-9]*-,,g" | sed -e "s,.gz,,g"`
  alt_xen_version=`echo $xen_version | sed -e "s,\.old$,,g"`

  list="/boot/vmlinuz-linux";

  while [ "x$list" != "x" ] ; do
    linux=`version_find_latest $list`
    echo -e "\tFound linux image: $linux" >&2
    basename=`basename $linux`
    dirname=`dirname $linux`
    rel_dirname=`make_system_path_relative_to_its_root $dirname`
    version=`echo $basename | sed -e "s,^[^0-9]*-,,g"`
    base_init=`echo $basename | sed -e "s,vmlinuz,initramfs,g"`
    alt_version="${base_init}-fallback"
    linux_root_device_thisversion="${LINUX_ROOT_DEVICE}"
    initrd=

    for i in "${base_init}.img"; do
       if test -e "${dirname}/${i}" ; then
         initrd="$i"
         break
       fi
    done
    if test -n "${initrd}" ; then
      echo -e "\tFound initrd image: ${dirname}/${initrd}" >&2
    else
      # "UUID=" magic is parsed by initrds.  Since there's no initrd, it can't work here.
      linux_root_device_thisversion=${GRUB_DEVICE}
    fi

    xen_entry "${OS}" "${xen_version}" "${version}" \
        "${XEN_HYPERVISOR_CMDLINE}" \
       "${XEN_LINUX_CMDLINE}"

    list=`echo $list | tr ' ' '\n' | grep -vx $linux | tr '\n' ' '`
  done

  xen_list=`echo $xen_list | tr ' ' '\n' | grep -vx $xen | tr '\n' ' '`
done
\end{lstlisting}

%\chapter{Obsah CD}
%\chapter{Manual}
%\chapter{Konfigrační soubor}
%\chapter{RelaxNG Schéma konfiguračního soboru}
%\chapter{Plakat}
 % viz. prilohy.tex
\end{document}
