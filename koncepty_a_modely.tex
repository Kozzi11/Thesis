
\section{Úvod}
Potřebu zaznamenávat informace (data) má lidstvo již od nepaměti. Jedním z prvních způsobů bylo vytesávaní znaků do kamenných desek. To bylo později nahrazeno zaznamenáváním 
na papirus ve formě svitků. Papirus časem nahradil pergamen a ten byl nahrazen papírem a místo svitků se začaly texty uchovávat ve formě volných listů případně svazků a knížek.

Uchovávaní informací je ale jen jedna část problemu. Jelikož k čemu by nám sloužily zaznamenaná data, kdybychom se vnich nedokázali orientovat, a neuměli v nich vyhledávat potřebné informace.
Proto vznikly různé způsoby, jak jednotlivé dokumenty strukturovat (číslovaní stránek (svitků), kapitoly, rejstříky\dots). 

Krom orientace v samotném dokumentu, je zapotřebí mít možnost dohledávat i jednotlivé dokumenty jako celky.
Pro tyto účely se využívá zařazení dokumentů do různých skupin (kategorií). K dělení do kategorií se pak využívá různých atributů, jimiž daný dokument disponuje.
Mezi tyto atributy patří například název, autor, téma, datum, typ dokumentu atd. Krom zařazení do kategorie se dá využít atributů i při vyhledávání či řazení dokumentu.

Předchozí odstavec si nejlépe představíme na příkladu kartotéky obvodního lékaře. Ten při hledání karty pacienta vyhledavá dle atributů uvedených v tabulce \ref{kartoteka}. Za předpokladu, že jsou jednotlivé karty pacientů rozděleny do zásuvek podle roku narození. A zásuvky jsou rozděleny do sekcím označené názvem měsíce narození. A tyto sekce se dále dělí na podsekce dle dne narození a ty jsou následně seřazeny abecdně podle přímení pacienta. Tak by postup vyhledání pacientovy karty mohl být následující.
\begin{table}[h!]
\begin{center}
\begin{tabular}{|l|r|}
\hline
    \bfseries attribut & \bfseries hodnota\\ \hline
    jméno pacienta & Daniel Kozák \\
    datum narození & 23.11.1988 \\
    adresa trvalého pobytu & Bratrušov 95 \\
    \hline
\end{tabular}
\caption{Informace o pacientovy využité pro nalezení jeho karty.}
\label{kartoteka}
\end{center}
\end{table}
\newpage
Lékař otevře zásuvku s označením ročník 1988. Zde se přesune na záložku s označením Listopad a vní se přesune na podzáložku s označením 23. Zde snadno vyhledá všechny pacienty s přímením Kozák a mezi nimy vybera pacienta s jménem Daniel a trvalým pobytem na adrese Bratrušov 95.\footnote{V tomto zjednodušeném příkladu předpokládám existenci pouze jednoho pacienta se stejným jménem, datem narození a adresou trvalého pobytu. V praxi by bylo na kartě uvedene ještě rodné číslo.}

V předchozím příkladě by nalezení pacientovy karty trvalo několik vteřin. Což se dá brát jako dostačující, ale co v případě kdy lékař dostane za úkol nalést v této kartotéce, karty všech pacientů s trvalým pobytem v Bratrušově. V tomto případě by lékaři nezbylo nic jiného, než projít karty všech pacientů a znich vybrat ty, které mají u trvalého pobytu uvedeno Bratrušov. Toto by v závislosti na počtu pacientů mohlo lékaří či jeho sestřičce zabrat desítky minut nebo i hodin. To už není ideální doba a jediný způsob, jak tento čas zkrátit, je využít víc lidí k prohledávání kartotéky (paralerní zpracování).

Revolucí zaznamenávání a zpracovaní informací je jejich ukládání v elektonické formě. Zde je papír a jeho obsah nahrazen souborem ležícím na paměťovém úložišti.Vyhledávaní jednotlivých dokumentů má nastarosti souborový systém.
\cite[s.~1--52]{korth:dbsc}

\section{Datové modely}
Základní součástí databáze je její datový model, ten popisuje základní strukturu dat, to jak jsou data uložena, uspořádána, jejich sémantiku, vztahy mezi daty a jejich integritní omezení. Datový model umožňuje popsat návrh databáze na její fyzické i logické úrovni.\cite[s.~945--975]{korth:dbsc}

Datových modelů existuje celá řada. Já se zde omezím jen na základní čtyři, které jsou uplatněny v této práci.

\subsection{Relační model}
Je nejrozšířenějším používaným datovým modelem dnešení doby a velká část součastných databázových systému je založena právě na tomto datovém modelu.

Relační model je založen na kolekcích tabulek, kterých se využíva jak pro reprezentaci dat, tak i pro vztahy mezi těmito daty. Každá tabulka má několik sloupců, kde každý sloupec (též označován jako atribut) má svůj unikátní název (identifikátor), typ a rozsah neboli doménu.
\subsection{Entity-relationship model}
Entity-relationship (E-R) datový model využívá kolekce jednoduchých objektů, nazývaných entity, a vztahů mezi těmito entitami. Entita je abstrakcí nějaké věci nebo objektu z reálného světa, která je jednoznačně odlišitelná od ostatních objektů. 

Entity-relationship datový model je široce využíván při databázovém návrhu. Výsledkem návrhu jsou diagramy, které se nazývají entity-relationship diagramy neboli ER diagramy (zkráceně \emph{ERD}). 
\subsection{Objektově orientovaný model}
\subsection{Semi-structured (Multivalue) model}
\section{Typy datbází}
\subsection{Relační databáze}
\subsection{Objektově-relační databáze}
\subsection{Objektové a NoSQL databáze}
